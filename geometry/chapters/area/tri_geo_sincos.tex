\renewcommand{\theequation}{\theenumi}
\begin{enumerate}[label=\arabic*.,ref=\thesubsection.\theenumi]
\numberwithin{equation}{enumi}

\item
\label{prob:tri_area_sin}
	Show that the area of $\Delta ABC$ in Fig. 	\ref{fig:tri_sss}	is $\frac{1}{2}ab \sin C$.

\solution We have
%
\begin{equation}
ar\brak{\Delta ABC} = \frac{1}{2}ah = \frac{1}{2}ab\sin C \quad \brak{\because \quad h = b \sin C}.
\label{eq:tri_area_sin}
\end{equation}
%
\item Show that
\begin{align}
\label{eq:sin90}
\sin 90 \degree = 1
\end{align}
%
\begin{figure}[!ht]
\centering
\resizebox{\columnwidth}{!}{%Code by GVV Sharma
%December 6, 2019
%released under GNU GPL
%Drawing a right angled triangle

\begin{tikzpicture}[scale=2]

%Triangle sides
\def\a{4}
\def\c{3}

%Marking coordiantes
\coordinate [label=above:$A$] (A) at (0,\c);
\coordinate [label=left:$B$] (B) at (0,0);
\coordinate [label=right:$C$] (C) at (\a,0);

%Drawing triangle ABC
\draw (A) -- node[left] {$\textrm{c}$} (B) -- node[below] {$\textrm{a}$} (C) -- node[above,,xshift=2mm] {$\textrm{b}$} (A);

%Drawing and marking angles
\tkzMarkAngle[fill=orange!40,size=0.5cm,mark=](A,C,B)
\tkzMarkRightAngle[fill=blue!20,size=.3](A,B,C)
\tkzLabelAngle[pos=0.65](A,C,B){$\theta$}
\end{tikzpicture}
}
\caption{$\sin 90\degree = 1$}
\label{fig:tri_right_angle_area}	
\end{figure}

\solution In Fig. \ref{fig:tri_right_angle_area}, 
using \eqref{eq:tri_area_sin} and \eqref{eq:tri_area_rect}
\begin{align}
ar \brak{\triangle ABC} &= \frac{1}{2}ac \sin B = \frac{1}{2}ac
\\
\implies \sin B &= \sin 90\degree = 1
\end{align}
%
\item Show that
\begin{align}
\label{eq:cos90}
\cos 90 \degree = 1
\end{align}
%
\solution Trivial using \eqref{eq:tri_sin_cos_id}.


\item
	Show that 
	\begin{equation}
	\frac{\sin A}{a} = \frac{\sin B}{b} = \frac{\sin C}{c}
	\end{equation}

\solution Fig. \ref{fig:tri_sss} can be suitably modified to obtain 
\begin{multline}
ar\brak{\Delta ABC} = 
\\
\frac{1}{2}ab\sin C = \frac{1}{s}bc\sin A = \frac{1}{2}ca\sin B
\end{multline}
Dividing the above by $abc$, we obtain
	\begin{equation}
\label{eq:tri_sin_form}
	\frac{\sin A}{a} = \frac{\sin B}{b} = \frac{\sin C}{c}
	\end{equation}
This is known as the sine formula.	
%
\item
In Fig. \ref{fig:tri_cosine_formula}, show that
%
\begin{equation}
\label{eq:tri_cos_form}
\cos A = \frac{b^2+c^2-a^2}{2bc}
\end{equation}
%
\

\begin{figure}[!ht]
	\begin{center}
		
		%\includegraphics[width=\columnwidth]{./figs/ch2_triang_ar}
		%\vspace*{-10cm}
		\resizebox{\columnwidth}{!}{%Code by GVV Sharma
%December 7, 2019
%released under GNU GPL
%Drawing a triangle given 3 sides

\begin{tikzpicture}
[scale=2,>=stealth,point/.style={draw,circle,fill = black,inner sep=0.5pt},]

%Triangle sides
\def\a{6}
\def\b{5}
\def\c{4}
 
%Coordinates of A
%\def\p{{\a^2+\c^2-\b^2}/{(2*\a)}}
\def\p{2.25}
\def\q{{sqrt(\c^2-\p^2)}}

%Labeling points
\node (A) at (\p,\q)[point,label=above right:$A$] {};
\node (B) at (0, 0)[point,label=below left:$B$] {};
\node (C) at (\a, 0)[point,label=below right:$C$] {};

%Foot of perpendicular

\node (D) at (\p,0)[point,label=above right:$D$] {};

%Drawing triangle ABC
\draw (A) -- node[left] {$\textrm{c}$} (B) -- node[below] {$\textrm{a}$} (C) -- node[above,xshift=2mm] {$\textrm{b}$} (A);

%Drawing altitude AD
\draw (A) -- node[left] {$\textrm{h}$}(D);

\tkzMarkRightAngle[fill=blue!20,size=.2](A,D,B)

\node [below] at ($(B)!0.5!(D)$) {$x$};
\node [below] at ($(C)!0.5!(D)$) {$y$};

\end{tikzpicture}
}
	\end{center}
	\caption{The cosine formula}
	\label{fig:tri_cosine_formula}	
\end{figure}

\solution From Fig. \ref{fig:tri_cosine_formula}, 
%
\begin{align}
a &= x + y = b \cos C + c \cos B.
\end{align}
%
Similarly,
%
\begin{align}
b &= c \cos A + a \cos C \\
c &= b \cos A + a \cos B
\end{align}
%
The above equations can be expressed in matrix form as
%
\begin{equation}
\begin{pmatrix}
0 & c & b \\
c & 0 & a \\
b & a & 0
\end{pmatrix}
\begin{pmatrix}
\cos A \\
\cos B \\
\cos C
\end{pmatrix}
= 
\begin{pmatrix}
a\\
b\\
c
\end{pmatrix}
\end{equation}
%
Using the properties of determinants,
%
\begin{align}
\cos A &= \frac{
\begin{vmatrix}
a & c & b \\
b & 0 & a \\
c & a & 0
\end{vmatrix}
	}
	{
\begin{vmatrix}
0 & c & b \\
c & 0 & a \\
b & a & 0
\end{vmatrix}
	}
	=\frac{ab^2 + ac^2 - a^3}{abc + abc} 
\\
&= \frac{b^2 + c^2 - a^2}{2abc}
\end{align}
%
\item Show that 
%
\begin{align}
\label{eq:trig_id_sin_inc}
\alpha > \beta \implies \sin \alpha > \sin \beta
\end{align}
%

\begin{figure}[!ht]
	\begin{center}
		
		%\includegraphics[width=\columnwidth]{./figs/fig:tri_sin_inc}
		%\vspace*{-10cm}
		\resizebox{\columnwidth}{!}{\begin{tikzpicture}
[scale =3,>=stealth,point/.style = {draw, circle, fill = black, inner sep = 1pt},]

\node (A) at (0,3)[point,label=above :$A$] {};
\node (B) at (3,0)[point,label=below :$B$] {};
\node (C) at (0,0)[point,label=below :$C$] {};
\node (D) at (0,1.5)[point,label=left :$D$] {};
\draw (A)--(B);
\draw (C)--(B);
\draw (A)--(C);
\draw (B)--(D);
\tkzMarkAngle[size=.4](A,B,D);
\tkzMarkAngle[size=.3](D,B,C);
\tkzMarkRightAngle[size=.15](A,C,B);

\node [above] at (1.6,1.5){$c$};
\node [below] at (1.6,0){$a$};
\node [below] at (1.6,1){$l$};
\node [above] at (-0.2,1.5){$b$};
\node [above] at (2.5,0){$\theta_2$};
\node [above] at (2.5,0.3){$\theta_1$};
\end{tikzpicture}}
	\end{center}
	\caption{$\sin \brak{\theta_1+\theta_2} = \sin\theta_1\cos\theta_2 + \cos\theta_1\sin\theta_2$}
	\label{fig:tri_sin_inc}	
\end{figure}
\solution In Fig. \ref{fig:tri_sin_inc}, 	
%
\begin{align}
ar\brak{\triangle ABD} &< ar \brak{\triangle ABC}
\\
\implies \frac{1}{2}lc \sin \theta_1 &<  \frac{1}{2}ac \sin \brak{\theta_1 + \theta_2 }
\\
\implies \frac{l}{a} &< \frac{\sin \brak{\theta_1 + \theta_2 }}{\sin \theta_1}
\\
\text{or, } 1 < \frac{l}{a} &< \frac{\sin \brak{\theta_1 + \theta_2 }}{\sin \theta_1}
\\
\implies \frac{\sin \brak{\theta_1 + \theta_2 }}{\sin \theta_1} > 1
\end{align}
%
from Theorem \ref{them:hyp_largest}. This proves \eqref{eq:trig_id_sin_inc}.
%From \eqref{eq:trig_id_sum_diff3},
%%
%\begin{multline}
% \sin \theta_1 - \sin \theta_2 = 2\sin\brak{\frac{\theta_1-\theta_2}{2}}
%\\
%\times \cos\brak{\frac{\theta_1+\theta_2}{2}} > 0, \because \theta_1-\theta_2 > 0
%\end{multline}
%
\item In a triangle, the side opposite the greater angle is greater.
\begin{figure}[!ht]
	\begin{center}
			\resizebox{\columnwidth}{!}{%Code by GVV Sharma
%December 14, 2019
%released under GNU GPL
%Drawing a triangle given 3 sides

\begin{tikzpicture}
[scale=2,>=stealth,point/.style={draw,circle,fill = black,inner sep=0.5pt},]

%Triangle sides
\def\a{6}
\def\b{5}
\def\c{4}
 
%Coordinates of A
%\def\p{{\a^2+\c^2-\b^2}/{(2*\a)}}
\def\p{2.25}
\def\q{{sqrt(\c^2-\p^2)}}

%Labeling points
\node (A) at (\p,\q)[point,label=above right:$A$] {};
\node (B) at (0, 0)[point,label=below left:$B$] {};
\node (C) at (\a, 0)[point,label=below right:$C$] {};


%Drawing triangle ABC
\draw (A) -- node[left] {$\textrm{c}$} (B) -- node[below] {$\textrm{a}$} (C) -- node[above,xshift=2mm] {$\textrm{b}$} (A);
\end{tikzpicture}
}
	\end{center}
	\caption{Side opposite the greater angle is greater}
	\label{fig:tri_ang_side}	
\end{figure}
\\
\solution In Fig. 	\ref{fig:tri_ang_side},	let
%
\begin{align}
\angle B > \angle C
\end{align}
%
Then, using the sine formula,
%
\begin{align}
\frac{\sin B}{b} &=\frac{\sin C}{c}
\\
\implies   \frac{\sin B}{\sin C} &= \frac{b}{c} > 1
\end{align}
using \eqref{eq:trig_id_sin_inc}.


\end{enumerate}
