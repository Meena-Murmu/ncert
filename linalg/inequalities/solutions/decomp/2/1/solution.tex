  Let 
  \begin{align}
\alpha=\myvec{1\\2}\\
\beta=\myvec{-1\\3}
\end{align}
We can express these as
\begin{align}
\alpha=k_1\vec{u}_1\label{eq:2.0.3}\\
\beta=r_1\vec{u}_1+k_2\vec{u}_2\label{eq:2.0.4}
\end{align}
where 
\begin{align}
k_1 = \norm{\alpha}\label{eq:2.0.5}\\\vec{u}_1 = \frac{\vec{\alpha}}{k_1} 
\\
r_1 = \frac{\vec{u}_1^T\beta}{\norm{\vec{u}_1}^2}\\
\vec{u}_2 = \frac{\beta - r_1 \vec{u}_1}{\norm{\beta - r_1 \vec{u}_1}}
\\
k_2 = {\vec{u}_2^T\beta}\label{eq:2.0.9}
\end{align}
From \eqref{eq:2.0.3} and \eqref{eq:2.0.4}, 
\begin{align}
\myvec{\alpha & \beta } = \myvec{\vec{u}_1 & \vec{u}_2}\myvec{k_1 & r_1 \\ 0 & k_2} \\
\myvec{\alpha & \beta } = \vec{Q}\vec{R}
\end{align}
From above we can see that $\vec{R}$ is an upper triangular matrix and
\begin{align}
\vec{Q}^T\vec{Q}=\vec{I}
\end{align}
Now by using equations \eqref{eq:2.0.5} to \eqref{eq:2.0.9} 
\begin{align}
k_1 = \sqrt{5}\\ \vec{u}_1 = \sqrt{\frac{1}{5}} \myvec{1\\2},
\\
r_1 = \sqrt{5}\\ \vec{u}_2 = \sqrt{\frac{1}{5}}\myvec{-2\\1}
\\
k_2 = \sqrt{5}
\end{align}
Thus obtained QR decomposition is
\begin{align}
\myvec{1&-1\\2&3}=\myvec{\frac{1}{\sqrt{5}}& -\frac{2}{\sqrt{5}}\\\frac{2}{\sqrt{5}}& \frac{1}{\sqrt{5}}}\myvec{\sqrt{5}&\sqrt{5}\\0&\sqrt{5}}
\end{align}
