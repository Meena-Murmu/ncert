\documentclass[journal,12pt,twocolumn]{IEEEtran}
%
\usepackage{setspace}
\usepackage{gensymb}
%\doublespacing
\singlespacing

%\usepackage{graphicx}
%\usepackage{amssymb}
%\usepackage{relsize}
\usepackage[cmex10]{amsmath}
%\usepackage{amsthm}
%\interdisplaylinepenalty=2500
%\savesymbol{iint}
%\usepackage{txfonts}
%\restoresymbol{TXF}{iint}
%\usepackage{wasysym}
\usepackage{amsthm}
\usepackage{iithtlc}
\usepackage{mathrsfs}
\usepackage{txfonts}
\usepackage{stfloats}
\usepackage{steinmetz}
\usepackage{bm}
\usepackage{cite}
\usepackage{cases}
\usepackage{subfig}
%\usepackage{xtab}
\usepackage{longtable}
\usepackage{multirow}
%\usepackage{algorithm}
%\usepackage{algpseudocode}
\usepackage{enumitem}
\usepackage{mathtools}
\usepackage{tikz}
\usepackage{circuitikz}
\usepackage{verbatim}
\usepackage{tfrupee}
\usepackage[breaklinks=true]{hyperref}
%\usepackage{stmaryrd}
\usepackage{tkz-euclide} % loads  TikZ and tkz-base
%\usetkzobj{all}
\usepackage{listings}
    \usepackage{color}                                            %%
    \usepackage{array}                                            %%
    \usepackage{longtable}                                        %%
    \usepackage{calc}                                             %%
    \usepackage{multirow}                                         %%
    \usepackage{hhline}                                           %%
    \usepackage{ifthen}                                           %%
  %optionally (for landscape tables embedded in another document): %%
    \usepackage{lscape}     
\usepackage{multicol}
\usepackage{chngcntr}
%\usepackage{enumerate}

%\usepackage{wasysym}
%\newcounter{MYtempeqncnt}
\DeclareMathOperator*{\Res}{Res}
%\renewcommand{\baselinestretch}{2}
\renewcommand\thesection{\arabic{section}}
\renewcommand\thesubsection{\thesection.\arabic{subsection}}
\renewcommand\thesubsubsection{\thesubsection.\arabic{subsubsection}}

\renewcommand\thesectiondis{\arabic{section}}
\renewcommand\thesubsectiondis{\thesectiondis.\arabic{subsection}}
\renewcommand\thesubsubsectiondis{\thesubsectiondis.\arabic{subsubsection}}

% correct bad hyphenation here
\hyphenation{op-tical net-works semi-conduc-tor}
\def\inputGnumericTable{}                                 %%

\lstset{
%language=C,
frame=single, 
breaklines=true,
columns=fullflexible
}
%\lstset{
%language=tex,
%frame=single, 
%breaklines=true
%}

\begin{document}
%


\newtheorem{theorem}{Theorem}[section]
\newtheorem{problem}{Problem}
\newtheorem{proposition}{Proposition}[section]
\newtheorem{lemma}{Lemma}[section]
\newtheorem{corollary}[theorem]{Corollary}
\newtheorem{example}{Example}[section]
\newtheorem{definition}[problem]{Definition}
%\newtheorem{thm}{Theorem}[section] 
%\newtheorem{defn}[thm]{Definition}
%\newtheorem{algorithm}{Algorithm}[section]
%\newtheorem{cor}{Corollary}
\newcommand{\BEQA}{\begin{eqnarray}}
\newcommand{\EEQA}{\end{eqnarray}}
\newcommand{\define}{\stackrel{\triangle}{=}}

\bibliographystyle{IEEEtran}
%\bibliographystyle{ieeetr}


\providecommand{\mbf}{\mathbf}
\providecommand{\pr}[1]{\ensuremath{\Pr\left(#1\right)}}
\providecommand{\qfunc}[1]{\ensuremath{Q\left(#1\right)}}
\providecommand{\sbrak}[1]{\ensuremath{{}\left[#1\right]}}
\providecommand{\lsbrak}[1]{\ensuremath{{}\left[#1\right.}}
\providecommand{\rsbrak}[1]{\ensuremath{{}\left.#1\right]}}
\providecommand{\brak}[1]{\ensuremath{\left(#1\right)}}
\providecommand{\lbrak}[1]{\ensuremath{\left(#1\right.}}
\providecommand{\rbrak}[1]{\ensuremath{\left.#1\right)}}
\providecommand{\cbrak}[1]{\ensuremath{\left\{#1\right\}}}
\providecommand{\lcbrak}[1]{\ensuremath{\left\{#1\right.}}
\providecommand{\rcbrak}[1]{\ensuremath{\left.#1\right\}}}
\theoremstyle{remark}
\newtheorem{rem}{Remark}
\newcommand{\sgn}{\mathop{\mathrm{sgn}}}
\providecommand{\abs}[1]{\left\vert#1\right\vert}
\providecommand{\res}[1]{\Res\displaylimits_{#1}} 
\providecommand{\norm}[1]{\left\lVert#1\right\rVert}
%\providecommand{\norm}[1]{\lVert#1\rVert}
\providecommand{\mtx}[1]{\mathbf{#1}}
\providecommand{\mean}[1]{E\left[ #1 \right]}
\providecommand{\fourier}{\overset{\mathcal{F}}{ \rightleftharpoons}}
%\providecommand{\hilbert}{\overset{\mathcal{H}}{ \rightleftharpoons}}
\providecommand{\system}{\overset{\mathcal{H}}{ \longleftrightarrow}}
	%\newcommand{\solution}[2]{\textbf{Solution:}{#1}}
\newcommand{\solution}{\noindent \textbf{Solution: }}
\newcommand{\cosec}{\,\text{cosec}\,}
\providecommand{\dec}[2]{\ensuremath{\overset{#1}{\underset{#2}{\gtrless}}}}
\newcommand{\myvec}[1]{\ensuremath{\begin{pmatrix}#1\end{pmatrix}}}
\newcommand{\mydet}[1]{\ensuremath{\begin{vmatrix}#1\end{vmatrix}}}
%\numberwithin{equation}{section}
\numberwithin{equation}{subsection}
%\numberwithin{problem}{section}
%\numberwithin{definition}{section}
\makeatletter
\@addtoreset{figure}{problem}
\makeatother

\let\StandardTheFigure\thefigure
\let\vec\mathbf
%\renewcommand{\thefigure}{\theproblem.\arabic{figure}}
\renewcommand{\thefigure}{\theproblem}
%\setlist[enumerate,1]{before=\renewcommand\theequation{\theenumi.\arabic{equation}}
%\counterwithin{equation}{enumi}


%\renewcommand{\theequation}{\arabic{subsection}.\arabic{equation}}

\def\putbox#1#2#3{\makebox[0in][l]{\makebox[#1][l]{}\raisebox{\baselineskip}[0in][0in]{\raisebox{#2}[0in][0in]{#3}}}}
     \def\rightbox#1{\makebox[0in][r]{#1}}
     \def\centbox#1{\makebox[0in]{#1}}
     \def\topbox#1{\raisebox{-\baselineskip}[0in][0in]{#1}}
     \def\midbox#1{\raisebox{-0.5\baselineskip}[0in][0in]{#1}}

\vspace{3cm}

\title{
	\logo{
Linear Forms
	}
}
\author{ G V V Sharma$^{*}$% <-this % stops a space
	\thanks{*The author is with the Department
		of Electrical Engineering, Indian Institute of Technology, Hyderabad
		502285 India e-mail:  gadepall@iith.ac.in. All content in this manual is released under GNU GPL.  Free and open source.}
	
}	
%\title{
%	\logo{Matrix Analysis through Octave}{\begin{center}\includegraphics[scale=.24]{tlc}\end{center}}{}{HAMDSP}
%}


% paper title
% can use linebreaks \\ within to get better formatting as desired
%\title{Matrix Analysis through Octave}
%
%
% author names and IEEE memberships
% note positions of commas and nonbreaking spaces ( ~ ) LaTeX will not break
% a structure at a ~ so this keeps an author's name from being broken across
% two lines.
% use \thanks{} to gain access to the first footnote area
% a separate \thanks must be used for each paragraph as LaTeX2e's \thanks
% was not built to handle multiple paragraphs
%

%\author{<-this % stops a space
%\thanks{}}
%}
% note the % following the last \IEEEmembership and also \thanks - 
% these prevent an unwanted space from occurring between the last author name
% and the end of the author line. i.e., if you had this:
% 
% \author{....lastname \thanks{...} \thanks{...} }
%                     ^------------^------------^----Do not want these spaces!
%
% a space would be appended to the last name and could cause every name on that
% line to be shifted left slightly. This is one of those "LaTeX things". For
% instance, "\textbf{A} \textbf{B}" will typeset as "A B" not "AB". To get
% "AB" then you have to do: "\textbf{A}\textbf{B}"
% \thanks is no different in this regard, so shield the last } of each \thanks
% that ends a line with a % and do not let a space in before the next \thanks.
% Spaces after \IEEEmembership other than the last one are OK (and needed) as
% you are supposed to have spaces between the names. For what it is worth,
% this is a minor point as most people would not even notice if the said evil
% space somehow managed to creep in.



% The paper headers
%\markboth{Journal of \LaTeX\ Class Files,~Vol.~6, No.~1, January~2007}%
%{Shell \MakeLowercase{\textit{et al.}}: Bare Demo of IEEEtran.cls for Journals}
% The only time the second header will appear is for the odd numbered pages
% after the title page when using the twoside option.
% 
% *** Note that you probably will NOT want to include the author's ***
% *** name in the headers of peer review papers.                   ***
% You can use \ifCLASSOPTIONpeerreview for conditional compilation here if
% you desire.




% If you want to put a publisher's ID mark on the page you can do it like
% this:
%\IEEEpubid{0000--0000/00\$00.00~\copyright~2007 IEEE}
% Remember, if you use this you must call \IEEEpubidadjcol in the second
% column for its text to clear the IEEEpubid mark.



% make the title area
\maketitle

\newpage

\tableofcontents

\bigskip

\renewcommand{\thefigure}{\theenumi}
\renewcommand{\thetable}{\theenumi}
%\renewcommand{\theequation}{\theenumi}

%\begin{abstract}
%%\boldmath
%In this letter, an algorithm for evaluating the exact analytical bit error rate  (BER)  for the piecewise linear (PL) combiner for  multiple relays is presented. Previous results were available only for upto three relays. The algorithm is unique in the sense that  the actual mathematical expressions, that are prohibitively large, need not be explicitly obtained. The diversity gain due to multiple relays is shown through plots of the analytical BER, well supported by simulations. 
%
%\end{abstract}
% IEEEtran.cls defaults to using nonbold math in the Abstract.
% This preserves the distinction between vectors and scalars. However,
% if the journal you are submitting to favors bold math in the abstract,
% then you can use LaTeX's standard command \boldmath at the very start
% of the abstract to achieve this. Many IEEE journals frown on math
% in the abstract anyway.

% Note that keywords are not normally used for peerreview papers.
%\begin{IEEEkeywords}
%Cooperative diversity, decode and forward, piecewise linear
%\end{IEEEkeywords}



% For peer review papers, you can put extra information on the cover
% page as needed:
% \ifCLASSOPTIONpeerreview
% \begin{center} \bfseries EDICS Category: 3-BBND \end{center}
% \fi
%
% For peerreview papers, this IEEEtran command inserts a page break and
% creates the second title. It will be ignored for other modes.
%\IEEEpeerreviewmaketitle

\begin{abstract}
This book provides a computational approach to school geometry based on the NCERT textbooks from Class 6-12.  Links to sample Python codes are available in the text.  
\end{abstract}
Download python codes using 
\begin{lstlisting}
svn co https://github.com/gadepall/school/trunk/ncert/computation/codes
\end{lstlisting}

%\section{Triangle}
\section{Examples}
\renewcommand{\theequation}{\theenumi}
\begin{enumerate}[label=\thesection.\arabic*.,ref=\thesection.\theenumi]
\numberwithin{equation}{enumi}

\renewcommand{\theequation}{\theenumi}
\begin{enumerate}[label=\arabic*.,ref=\thesection.\theenumi]
\numberwithin{equation}{enumi}
%
%
\item Express the problem of finding the distance of the point $\vec{P}=\myvec{3\\-5}$ from the line 
\label{prob:opt_line_dist}
\begin{align}
\label{eq:opt_line_nor}
L: \quad \myvec{3 & – 4}\vec{x}  = 26
\end{align}
%
as an optimization problem.
\\
\solution The given problem can be expressed as
%
\begin{align}
\label{eq:opt_line_dist}
\min_{\vec{x}}g(\vec{x}) &= \norm{\vec{x}-\vec{P}}^2
\\
\text{s.t.} \quad \vec{n}^T\vec{x} &= c
\label{eq:opt_line_dist_nor}
\end{align}
%
where 
%
\begin{align}
\vec{n} &= \myvec{3\\-4}
\\
c&=26
\end{align}
%
\item Explain Problem \ref{prob:opt_line_dist} through a plot and find a graphical solution.
%
\item Solve \eqref{eq:opt_line_dist} using cvxpy.
%
\\
\solution  The following code yields
%	
\begin{lstlisting}
codes/line_dist_cvx.py
\end{lstlisting}
%
\begin{align}
\vec{x}_{\min} &= \myvec{2.64\\-4.52},
\\
g\brak{\vec{x}_{\min}} &= 0.6
\end{align}
%

\item Convert  \eqref{eq:opt_line_dist} to an {\em unconstrained} optimization problem.
%
\\
%
\solution $L$ in \eqref{eq:opt_line_nor} can be expressed in terms of the direction vector $\vec{m}$ as
\begin{align}
\label{eq:opt_line_dir}
\vec{x} = \vec{A} + \lambda \vec{m}, 
\end{align}
where $\vec{A}$ is any point on the line and 
%
\begin{align}
\label{eq:opt_line_orth}
\vec{m}^T\vec{n} = 0
\end{align}
%
Substituting \eqref{eq:opt_line_dir} in \eqref{eq:opt_line_dist}, an unconstrained optimization problem 
\begin{align}
\label{eq:opt_line_dist_uncon}
\min_{\lambda}f(\lambda) = \norm{\vec{A} + \lambda \vec{m}-\vec{P}}^2 
\end{align}
%
is obtained.
%
\item Solve \eqref{eq:opt_line_dist_uncon}.
%
\\
\solution 
\begin{align}
f(\lambda) 
& = \brak{ \lambda \vec{m}+\vec{A} -\vec{P}}^T \brak{ \lambda \vec{m}+\vec{A} -\vec{P}}
\\
&= \lambda^2 \norm{\vec{m}}^2 +2\lambda \vec{m}^T\brak{\vec{A} -\vec{P}} 
\nonumber \\
&\quad + \norm{\vec{A} -\vec{P}}^2
\label{eq:opt_line_dist_uncon_dist}
\end{align}
\begin{align}
\because f^{(2)}\lambda = 2\norm{\vec{m}}^2 > 0
\end{align}
%
the minimum value of $f(\lambda)$ is obtained when 
%
\begin{align}
 f^{(1)}(\lambda) &= 2\lambda\norm{\vec{m}}^2 + 2 \vec{m}^T\brak{\vec{A} -\vec{P}} =0
\\
\implies \lambda_{\min} &= -\frac{\vec{m}^T\brak{\vec{A} -\vec{P}}}{\norm{\vec{m}}^2}
\label{eq:opt_line_dist_uncon_lam_min}
\end{align}
%
Choosing $\vec{A}$ such that 
%
\begin{align}
\vec{m}^T\brak{\vec{A} -\vec{P}} &= 0,
\label{eq:opt_line_dist_uncon_trick}
\end{align}
%
substituting in \eqref{eq:opt_line_dist_uncon_lam_min},
%
\begin{align}
\label{eq:opt_line_dist_uncon_lam0}
\lambda_{\min} &= 0 \quad \text{and}
\\
\vec{A} -\vec{P} &= \mu \vec{n}
\label{eq:opt_line_dist_uncon_mu}
\end{align}
for some constant $\mu$. \eqref{eq:opt_line_dist_uncon_mu}
 is a consequence of \eqref{eq:opt_line_orth} and \eqref{eq:opt_line_dist_uncon_trick}. Also, from 
\eqref{eq:opt_line_dist_uncon_mu},
%
\begin{align}
\vec{n}^T\brak{\vec{A} -\vec{P} } &= \mu \norm{\vec{n}}^2
\\
\implies \mu & = \frac{\vec{n}^T\vec{A} -\vec{n}^T\vec{P} }{\norm{\vec{n}}^2} = \frac{c -\vec{n}^T\vec{P} }{\norm{\vec{n}}^2}
\label{eq:opt_line_dist_uncon_mu_sol}
\end{align}
%
from \eqref{eq:opt_line_dist_nor}.
%, where $\mu$ is some constant.
Substituting $\lambda_{\min} = 0$ in \eqref{eq:opt_line_dist_uncon},
%\label
%Thus, the shortest distance from $\vec{P}$ to $L$ is
%
\begin{align}
\min_{\lambda}f(\lambda) =  \norm{\vec{A} -\vec{P}}^2 = \mu^2\norm{\vec{n}}^2
\label{eq:opt_line_dist_uncon_f}
\end{align}
upon substituting from \eqref{eq:opt_line_dist_uncon_mu}. The distance between $\vec{P}$ and ${L}$ is then obtained from \eqref{eq:opt_line_dist_uncon_f} as
% obtained as 
\begin{align}
\norm{\vec{A} -\vec{P}} &= \abs{\mu}\norm{\vec{n}}
\\
&= \frac{\abs{\vec{n}^T\vec{P} -c }}{\norm{\vec{n}}}
\label{eq:opt_line_dist_uncon_f_closed}
\end{align}
after substituting for $\mu$ from  \eqref{eq:opt_line_dist_uncon_mu_sol}. Using the corresponding values from Problem \eqref{prob:opt_line_dist} in \eqref{eq:opt_line_dist_uncon_f_closed},
%
\begin{align}
\min_{\lambda}f(\lambda) =  0.6
\end{align}

%
%
\end{enumerate}


\end{enumerate}

\section{Exercises}
\renewcommand{\theequation}{\theenumi}
\begin{enumerate}[label=\thesection.\arabic*.,ref=\thesection.\theenumi]
\numberwithin{equation}{enumi}
%\renewcommand{\theequation}{\theenumi}
%\begin{enumerate}[label=\arabic*.,ref=\thesubsection.\theenumi]
%\numberwithin{equation}{enumi}
%
%
\item Check which of the following are solutions of the equation 
%
\begin{align}
\myvec{1 & -2}\vec{x} &= 4
\end{align}
%
%
\begin{enumerate}[itemsep=2pt]
\begin{multicols}{2}
\item $\myvec{0 \\ 2}$
\item $\myvec{2 \\ 0}$
\item $\myvec{4 \\ 0}$
\item $\myvec{\sqrt{2} \\ 4\sqrt{2}}$
\item $\myvec{1 \\ 1}$
\end{multicols}
\end{enumerate}
%
\item Find the value of $k$, if $\myvec{2\\1}$ is a solution of the equation 
%
%
\begin{align}
\myvec{2 & 3}\vec{x} &= k
\end{align}
%
%
\item Draw the graphs of the following equations
\begin{enumerate}[itemsep=2pt]
%\begin{multicols}{2}
\item $\myvec{1 & 1}\vec{x} = 4$
\item $\myvec{ 1 & -1}\vec{x}  = 2 $
\item $\myvec{ 3 & -1}\vec{x}  = 0$
\item $\myvec{ 2 & 1}\vec{x}  = 3$
\item $\myvec{ 1 & -1}\vec{x}  = 0$
\item $\myvec{ 1& 1}\vec{x}  = 0$
\item $\myvec{ 2& -1}\vec{x}  = 0$
\item $\myvec{ 7& -3}\vec{x}  = 2$
\item $\myvec{ 1& 1}\vec{x}  = 0$
\item $\myvec{ 1& -1}\vec{x}  = -2$
\item $\myvec{ 1& 1}\vec{x}  = 2$
\item $\myvec{ 1& 2}\vec{x}  = 6$
%\end{multicols}
\end{enumerate}
%
\item Give the equations of two lines passing through \myvec{2 \\ 14}. How many more such lines are there, and why?
\item If the point \myvec{3 \\ 4} lies on the graph of the equation $3y = ax + 7$, find the value of $a$
\item Find out whether the lines representing the
following pairs of linear equations intersect at a point, are parallel or coincident
%
\begin{enumerate}[itemsep=2pt]
%\begin{multicols}{2}
\item
\begin{align}
\begin{split}
\myvec{5 & -4 }\vec{x}&=-8
\\
\myvec{7 & 6 }\vec{x}&=9
\end{split}
\end{align}
\item
\begin{align}
\begin{split}
\myvec{9 & 3 }\vec{x}&=-12
\\
\myvec{18 & 6 }\vec{x}&=-24
\end{split}
\end{align}
\item
\begin{align}
\begin{split}
\myvec{6 & -3 }\vec{x}&=-10
\\
\myvec{2 & -1 }\vec{x}&=-9
\end{split}
\end{align}
%\end{multicols}
\end{enumerate}
%
\item Find out whether the following pair of linear
equations are consistent, or inconsistent.
%
\begin{enumerate}[itemsep=2pt]
%\begin{multicols}{2}
\item
\begin{align}
\begin{split}
\myvec{3 & 2 }\vec{x}&=5
\\
\myvec{2 & -3 }\vec{x}&=7
\end{split}
\end{align}
\item
\begin{align}
\begin{split}
\myvec{2 & -3 }\vec{x}&=8
\\
\myvec{4 & -6 }\vec{x}&=9
\end{split}
\end{align}
\item
\begin{align}
\begin{split}
\myvec{\frac{3}{2} & \frac{5}{3} }\vec{x}&=7
\\
\myvec{9 & -10 }\vec{x}&=14
\end{split}
\end{align}
\item
\begin{align}
\begin{split}
\myvec{5 & -3 }\vec{x}&=11
\\
\myvec{-10 & 6 }\vec{x}&=-22
\end{split}
\end{align}
\item
\begin{align}
\begin{split}
\myvec{\frac{4}{3} & 2 }\vec{x}&=8
\\
\myvec{2 & 3 }\vec{x}&=12
\end{split}
\end{align}
%\end{multicols}
\end{enumerate}
%
\item Which of the following pairs of linear equations are consistent/inconsistent? If consistent, obtain the solution:
%
\begin{enumerate}[itemsep=2pt]
%\begin{multicols}{2}
\item
\begin{align}
\begin{split}
\myvec{1 & 1 }\vec{x}&=5
\\
\myvec{2 & 2 }\vec{x}&=10
\end{split}
\end{align}
\item
\begin{align}
\begin{split}
\myvec{1 & -1 }\vec{x}&=8
\\
\myvec{3 & -3 }\vec{x}&=16
\end{split}
\end{align}
\item
\begin{align}
\begin{split}
\myvec{2 & 1 }\vec{x}&=6
\\
\myvec{4 & -2 }\vec{x}&=4
\end{split}
\end{align}
\item
\begin{align}
\begin{split}
\myvec{2 & -2 }\vec{x}&=2
\\
\myvec{4 & -4 }\vec{x}&=5
\end{split}
\end{align}
%\end{multicols}
\end{enumerate}
%
\item Given the linear equation $\myvec{2 & 3}\vec{x} – 8 = 0$, write another linear equation in two variables such that the geometrical representation of the pair so formed is: 
%
\begin{enumerate}[itemsep=2pt]
\begin{multicols}{2}
\item  intersecting lines
\item parallel lines 
\item  coincident lines
\end{multicols}
\end{enumerate}
%
%
\item Find the intersection of the following lines
%
\begin{enumerate}[itemsep=2pt]
%\begin{multicols}{2}
\item
\begin{align}
\begin{split}
\myvec{1 & 1 }\vec{x}&=14
\\
\myvec{1 & -1 }\vec{x}&=4
\end{split}
\end{align}
\item
\begin{align}
\begin{split}
\myvec{1 & -1 }\vec{x}&=3
\\
\myvec{\frac{1}{3} & \frac{1}{2} }\vec{x}&=6
\end{split}
\end{align}
\item
\begin{align}
\begin{split}
\myvec{3 & -1 }\vec{x}&=3
\\
\myvec{9 & -3 }\vec{x}&=9
\end{split}
\end{align}
\item
\begin{align}
\begin{split}
\myvec{0.2 & 0.3 }\vec{x}&=1.3
\\
\myvec{0.4 & 0.5 }\vec{x}&=2.3
\end{split}
\end{align}
\item
\begin{align}
\begin{split}
\myvec{\sqrt{2} & \sqrt{3} }\vec{x}&=0
\\
\myvec{\sqrt{3} & \sqrt{8} }\vec{x}&=0
\end{split}
\end{align}
\item
\begin{align}
\begin{split}
\myvec{\frac{3}{2} & -\frac{5}{3} }\vec{x}&=-2
\\
\myvec{\frac{1}{3} & \frac{1}{2} }\vec{x}&=\frac{13}{6}
\end{split}
\end{align}
%\end{multicols}
\end{enumerate}
%
%
%
\item Which of the following pairs of linear equations has a unique solution, no solution, or infinitely many solutions?
%
\begin{enumerate}[itemsep=2pt]
%\begin{multicols}{2}
\item
\begin{align}
\begin{split}
\myvec{1 & -3 }\vec{x}&=3
\\
\myvec{3 & -9 }\vec{x}&=2
\end{split}
\end{align}
\item
\begin{align}
\begin{split}
\myvec{2 & 1 }\vec{x}&=5
\\
\myvec{3 & 2 }\vec{x}&=8
\end{split}
\end{align}
\item
\begin{align}
\begin{split}
\myvec{3 & -5 }\vec{x}&=20
\\
\myvec{6 & -10 }\vec{x}&=40
\end{split}
\end{align}
\item
\begin{align}
\begin{split}
\myvec{1 & -3 }\vec{x}&=7
\\
\myvec{3 & -3 }\vec{x}&=15
\end{split}
\end{align}
%\end{multicols}
\end{enumerate}
%

\item Find the slope of the line, which makes an angle of $30\degree$ of y-axis measured anticlockwise.
\item Write the equations for the x and y axes.
\item Find the equation of the line satisfying the following conditions 
\begin{enumerate}
\item passing through  the point \myvec{-4\\3} with slope $\frac{1}{2}$.
\item passing through the point \myvec{0\\0} with slope $m$.
\item passing through the point $\myvec{2\\2\sqrt{3}}$ and inclined with the x-axis at an angle of 75$\degree$.
\item Intersecting the x-axis at a distance of 3 units to the let of the origin with slope -2.
\item intersecting the y-axis at a distance of 2 units above the origin and making an angle of $30\degree$ with the positive direction of the x-axis.
\item passing through the points \myvec{-1\\1} and \myvec{2\\-4}.
\item perpendicular distance from the origin is 5 and the angle made by the perpendicular with the positive x-axis is 30$\degree$.
\end{enumerate}
\item Find the equation of the line passing through \myvec{-3\\5} and perpendicular to the line through the points \myvec{2\\5} and \myvec{-3\\6}.
\item Find the direction vectors and and y-intercepts  of the following lines 
\begin{enumerate}
\item $\myvec{1 & 7}\vec{x} = 0$.
\item $\myvec{6 & 3}\vec{x} = 5$.
\item $\myvec{0 & 1}\vec{x} = 0$.
\end{enumerate}

\item Find the perpendicular distances of the following lines from the origin and angle between the perpendicular and the positive x-axis.
\begin{enumerate}
\item $\myvec{1 & -\sqrt{3}}\vec{x} = -8$.
\item $\myvec{0 & 1}\vec{x} = 2$.
\item $\myvec{1 & -1}\vec{x} = 4$.
\end{enumerate}
\item Find the distance between the parallel lines
%
\begin{align}
\myvec{15 & 8}\vec{x} &= 34
\\
\myvec{15 & 8}\vec{x} &= -31
\end{align}
\item Find the equation of the line parallel to the line 
\begin{align}
\myvec{3 & -4}\vec{x} = -2
\end{align}
%
and passing through the point \myvec{-2\\3}.
\item Find the alue of $p$ so that the three lines 
%
\begin{align}
\myvec{3 & 1}\vec{x} &= 2
\\
\myvec{p & 2}\vec{x} &= 3
\\
\myvec{2 & -1}\vec{x} &= 3
\end{align}
%
may intersect at one point.
%
%
\item The hypotenuse of a right angled triangle has its ends at the points \myvec{1\\3} and \myvec{-4\\1}. Find an equation of the legs of the triangle.

\item If the lines
%
%
\begin{align}
\myvec{-3 & 1}\vec{x} &= 1
\\
\myvec{-1 & 2}\vec{x} &= 3
\end{align}
%
are equally inclined to the line
%
\begin{align}
\myvec{-m & 1}\vec{x} &= 4,
\end{align}
%
find the value of $m$.
%
\item The sum of the perpendicular distances of a variable point $\vec{P}$ from the lines
%
\begin{align}
\myvec{1 & 1}\vec{x} &= 0
\\
\myvec{3 & -2}\vec{x} &= -7
\end{align}
%
is always 10.  Show that $\vec{P}$ must move on a line.
%
\item Find the equation of the line which is equidistant from parallel lines
%
\begin{align}
\myvec{9 & 7}\vec{x} &= 7
\\
\myvec{3 & 2}\vec{x} &= -6.
\end{align}
%
\item Determine the ratio in which the line 
\begin{align}
\myvec{2 & 1}\vec{x} - 4 = 0
\end{align}
%
divides the line segment joining the points $\vec{A}=\myvec{2\\-2}, \vec{B}=\myvec{3\\7}$.
\item A line perpendicular to the line segment joining the points \myvec{1\\0} and \myvec{2\\3} divides it in the ratio $1:n$.  Find the equation of the line.
\item Find the equation of a line that cuts off equal intercepts on the coordinate axes and passes through the point \myvec{2\\3}.
\item Find the equation of the line passing through the point \myvec{2\\2} and cutting off intercepts on the axes whose sum is 9.
\item Find the equation of the line through the point \myvec{0\\2} making an angle $\frac{2\pi}{3}$ with the positive x-axis.  Also, find the equation of the line parallel to it and crossing the y-axis at a distance of 2 units below the origin.
\item The perpendicular from the origin to a line meets it at a point \myvec{-2\\9}, find the equation of the line.
\item Find the angle between the following pair of lines:
\begin{align}
L_1: \quad \vec{x} &= \myvec{3\\1\\-2} + \lambda_1\myvec{1 \\ -1 \\-2}
\\
L_2: \quad \vec{x} &= \myvec{2\\-1\\-56} + \lambda_2\myvec{3 \\ -5 \\-4}
\end{align}
%\end{enumerate}
\item Find the shortest distance between the lines 
\begin{align}
L_1: \quad \vec{x} &= \myvec{1\\2\\1} + \lambda_1\myvec{1 \\ -1 \\1}
\\
L_2: \quad \vec{x} &= \myvec{2\\-1\\-1} + \lambda_2\myvec{2 \\ 1 \\2}
\end{align}
\item Find the shortest distance between the lines 
\begin{align}
\frac{x+1}{7} = \frac{y+1}{-6} &= \frac{z+1}{1}, 
\\
\frac{x-3}{1} = \frac{y-5}{-2} &= \frac{z-7}{1} 
\end{align}
%
\item Find the shortest distance between the lines 
\begin{align}
L_1: \quad \vec{x} &= \myvec{1\\2\\3} + \lambda_1\myvec{1 \\ -3 \\2}
\\
L_2: \quad \vec{x} &= \myvec{4\\5\\6} + \lambda_2\myvec{2 \\ 3 \\1}
\end{align}
%
\item Find the equation of the planes
\begin{enumerate}
\item that passes through the point \myvec{1\\0\\-2} and the normal to the plane is \myvec{1\\1\\-1}.
\item that passes through the point \myvec{1\\4\\6} and the normal vetor the plane is \myvec{1\\-2\\1}.
\end{enumerate}
\item Find the equation of the planes that passes through three points
\begin{enumerate}
\item \myvec{1\\1\\-1}, \myvec{6\\4\\-5}, \myvec{-4\\-2\\3}
\item \myvec{1\\1\\0}, \myvec{1\\2\\1}, \myvec{-2\\2\\-1}.
\end{enumerate}
\item Find the intercepts cut off by the plane 
$
\myvec{2 & 1 & 1}\vec{x}=5.
$
\item Find the equaion of the plane with intercept 3 on the y-axis and parallel to ZOX plane.
\item Find the equation of the plane through the intersection of the planes 
$
\myvec{3 & -1 & 2}\vec{x}=4
$
 and 
$
\myvec{1 & 1 & 1}\vec{x}=-2
$
and the pont \myvec{2\\2\\1}.
%
\item Find the equation of the plane passing through the intersection of the planes 
$
\myvec{2 & 2 & -3}\vec{x}=7
$
 and 
$
\myvec{2 & 5 & 3}\vec{x}=9
$
and the pont \myvec{2\\1\\3}.
%
\item  Find the equation of the plane through the intersection of the planes
$
\myvec{1 & 1 & 1}\vec{x}=1
$
 and 
$
\myvec{2 & 3 & 4}\vec{x}=5
$
which is perpendicular to the plane 
$
\myvec{1 & -1 & 1}\vec{x}=0.
$
%
\item Find the angle between the planes whose equations are
$
\myvec{2 & 2 & -3}\vec{x}=5
$
 and 
$
\myvec{3 & -3 & 5}\vec{x}=3
$
%
\item In the following cases, determine whether the given planes are parallel or perpendicular, and in case they are neither, find the angles between them.
\begin{enumerate}
\item 
$
\myvec{7 & 5 & 6}\vec{x}=-30
$
 and 
$
\myvec{3 & -1 & -10}\vec{x}=-4
$
%
\item 
$
\myvec{2 & 1 & 3}\vec{x}=2
$
 and 
$
\myvec{1 & -2 & 5}\vec{x}=0
$
%
\item 
$
\myvec{2 & -2 & 4}\vec{x}=-5
$
 and 
$
\myvec{3 & -3 & 6}\vec{x}=1
$
\item 
$
\myvec{2 & -1 & 3}\vec{x}=1
$
 and 
$
\myvec{2 & -1 & 3}\vec{x}=-3
$
\item 
$
\myvec{4 & 8 & 1}\vec{x}=8
$
 and 
$
\myvec{0 & 1 & 1}\vec{x}=4
$
\end{enumerate}
\item In the following cases, find the distance of each of the given points from the corresponding plane.
%\newcounter{rowno}
%\setcounter{rowno}{0}
\begin{table}[!h]
\centering
%\begin{tabular}{>{\stepcounter{rowno}\therowno.}cl}
%\multicolumn{1}{r}{No.} & text & abcd\\\hline
% & first  \\
% & second \\
% & third  \\
% & fourth 
%\end{tabular}
%%%%%%%%%%%%%%%%%%%%%%%%%%%%%%%%%%%%%%%%%%%%%%%%%%%%%%%%%%%%%%%%%%%%%%
%%                                                                  %%
%%  This is the header of a LaTeX2e file exported from Gnumeric.    %%
%%                                                                  %%
%%  This file can be compiled as it stands or included in another   %%
%%  LaTeX document. The table is based on the longtable package so  %%
%%  the longtable options (headers, footers...) can be set in the   %%
%%  preamble section below (see PRAMBLE).                           %%
%%                                                                  %%
%%  To include the file in another, the following two lines must be %%
%%  in the including file:                                          %%
%%        \def\inputGnumericTable{}                                 %%
%%  at the beginning of the file and:                               %%
%%        \input{name-of-this-file.tex}                             %%
%%  where the table is to be placed. Note also that the including   %%
%%  file must use the following packages for the table to be        %%
%%  rendered correctly:                                             %%
%%    \usepackage[latin1]{inputenc}                                 %%
%%    \usepackage{color}                                            %%
%%    \usepackage{array}                                            %%
%%    \usepackage{longtable}                                        %%
%%    \usepackage{calc}                                             %%
%%    \usepackage{multirow}                                         %%
%%    \usepackage{hhline}                                           %%
%%    \usepackage{ifthen}                                           %%
%%  optionally (for landscape tables embedded in another document): %%
%%    \usepackage{lscape}                                           %%
%%                                                                  %%
%%%%%%%%%%%%%%%%%%%%%%%%%%%%%%%%%%%%%%%%%%%%%%%%%%%%%%%%%%%%%%%%%%%%%%



%%  This section checks if we are begin input into another file or  %%
%%  the file will be compiled alone. First use a macro taken from   %%
%%  the TeXbook ex 7.7 (suggestion of Han-Wen Nienhuys).            %%
\def\ifundefined#1{\expandafter\ifx\csname#1\endcsname\relax}


%%  Check for the \def token for inputed files. If it is not        %%
%%  defined, the file will be processed as a standalone and the     %%
%%  preamble will be used.                                          %%
\ifundefined{inputGnumericTable}

%%  We must be able to close or not the document at the end.        %%
	\def\gnumericTableEnd{\end{document}}


%%%%%%%%%%%%%%%%%%%%%%%%%%%%%%%%%%%%%%%%%%%%%%%%%%%%%%%%%%%%%%%%%%%%%%
%%                                                                  %%
%%  This is the PREAMBLE. Change these values to get the right      %%
%%  paper size and other niceties.                                  %%
%%                                                                  %%
%%%%%%%%%%%%%%%%%%%%%%%%%%%%%%%%%%%%%%%%%%%%%%%%%%%%%%%%%%%%%%%%%%%%%%

	\documentclass[12pt%
			  %,landscape%
                    ]{report}
       \usepackage[latin1]{inputenc}
       \usepackage{fullpage}
       \usepackage{color}
       \usepackage{array}
       \usepackage{longtable}
       \usepackage{calc}
       \usepackage{multirow}
       \usepackage{hhline}
       \usepackage{ifthen}

	\begin{document}


%%  End of the preamble for the standalone. The next section is for %%
%%  documents which are included into other LaTeX2e files.          %%
\else

%%  We are not a stand alone document. For a regular table, we will %%
%%  have no preamble and only define the closing to mean nothing.   %%
    \def\gnumericTableEnd{}

%%  If we want landscape mode in an embedded document, comment out  %%
%%  the line above and uncomment the two below. The table will      %%
%%  begin on a new page and run in landscape mode.                  %%
%       \def\gnumericTableEnd{\end{landscape}}
%       \begin{landscape}


%%  End of the else clause for this file being \input.              %%
\fi

%%%%%%%%%%%%%%%%%%%%%%%%%%%%%%%%%%%%%%%%%%%%%%%%%%%%%%%%%%%%%%%%%%%%%%
%%                                                                  %%
%%  The rest is the gnumeric table, except for the closing          %%
%%  statement. Changes below will alter the table's appearance.     %%
%%                                                                  %%
%%%%%%%%%%%%%%%%%%%%%%%%%%%%%%%%%%%%%%%%%%%%%%%%%%%%%%%%%%%%%%%%%%%%%%

\providecommand{\gnumericmathit}[1]{#1} 
%%  Uncomment the next line if you would like your numbers to be in %%
%%  italics if they are italizised in the gnumeric table.           %%
%\renewcommand{\gnumericmathit}[1]{\mathit{#1}}
\providecommand{\gnumericPB}[1]%
{\let\gnumericTemp=\\#1\let\\=\gnumericTemp\hspace{0pt}}
 \ifundefined{gnumericTableWidthDefined}
        \newlength{\gnumericTableWidth}
        \newlength{\gnumericTableWidthComplete}
        \newlength{\gnumericMultiRowLength}
        \global\def\gnumericTableWidthDefined{}
 \fi
%% The following setting protects this code from babel shorthands.  %%
 \ifthenelse{\isundefined{\languageshorthands}}{}{\languageshorthands{english}}
%%  The default table format retains the relative column widths of  %%
%%  gnumeric. They can easily be changed to c, r or l. In that case %%
%%  you may want to comment out the next line and uncomment the one %%
%%  thereafter                                                      %%
\providecommand\gnumbox{\makebox[0pt]}
%%\providecommand\gnumbox[1][]{\makebox}

%% to adjust positions in multirow situations                       %%
\setlength{\bigstrutjot}{\jot}
\setlength{\extrarowheight}{\doublerulesep}

%%  The \setlongtables command keeps column widths the same across  %%
%%  pages. Simply comment out next line for varying column widths.  %%
\setlongtables

\setlength\gnumericTableWidth{%
	44pt+%
	44pt+%
	84pt+%
0pt}
\def\gumericNumCols{3}
\setlength\gnumericTableWidthComplete{\gnumericTableWidth+%
         \tabcolsep*\gumericNumCols*2+\arrayrulewidth*\gumericNumCols}
\ifthenelse{\lengthtest{\gnumericTableWidthComplete > \linewidth}}%
         {\def\gnumericScale{\ratio{\linewidth-%
                        \tabcolsep*\gumericNumCols*2-%
                        \arrayrulewidth*\gumericNumCols}%
{\gnumericTableWidth}}}%
{\def\gnumericScale{1}}

%%%%%%%%%%%%%%%%%%%%%%%%%%%%%%%%%%%%%%%%%%%%%%%%%%%%%%%%%%%%%%%%%%%%%%
%%                                                                  %%
%% The following are the widths of the various columns. We are      %%
%% defining them here because then they are easier to change.       %%
%% Depending on the cell formats we may use them more than once.    %%
%%                                                                  %%
%%%%%%%%%%%%%%%%%%%%%%%%%%%%%%%%%%%%%%%%%%%%%%%%%%%%%%%%%%%%%%%%%%%%%%

\ifthenelse{\isundefined{\gnumericColA}}{\newlength{\gnumericColA}}{}\settowidth{\gnumericColA}{\begin{tabular}{@{}p{44pt*\gnumericScale}@{}}x\end{tabular}}
\ifthenelse{\isundefined{\gnumericColB}}{\newlength{\gnumericColB}}{}\settowidth{\gnumericColB}{\begin{tabular}{@{}p{44pt*\gnumericScale}@{}}x\end{tabular}}
\ifthenelse{\isundefined{\gnumericColC}}{\newlength{\gnumericColC}}{}\settowidth{\gnumericColC}{\begin{tabular}{@{}p{84pt*\gnumericScale}@{}}x\end{tabular}}


\begin{tabular}[c]{%
	b{\gnumericColA}%
	b{\gnumericColB}%
	b{\gnumericColC}%
	}

%%%%%%%%%%%%%%%%%%%%%%%%%%%%%%%%%%%%%%%%%%%%%%%%%%%%%%%%%%%%%%%%%%%%%%
%%  The longtable options. (Caption, headers... see Goosens, p.124) %%
%	\caption{The Table Caption.}             \\	%
% \hline	% Across the top of the table.
%%  The rest of these options are table rows which are placed on    %%
%%  the first, last or every page. Use \multicolumn if you want.    %%

%%  Header for the first page.                                      %%
%	\multicolumn{3}{c}{The First Header} \\ \hline 
%	\multicolumn{1}{c}{colTag}	%Column 1
%	&\multicolumn{1}{c}{colTag}	%Column 2
%	&\multicolumn{1}{c}{colTag}	\\ \hline %Last column
%	\endfirsthead

%%  The running header definition.                                  %%
%	\hline
%	\multicolumn{3}{l}{\ldots\small\slshape continued} \\ \hline
%	\multicolumn{1}{c}{colTag}	%Column 1
%	&\multicolumn{1}{c}{colTag}	%Column 2
%	&\multicolumn{1}{c}{colTag}	\\ \hline %Last column
%	\endhead

%%  The running footer definition.                                  %%
%	\hline
%	\multicolumn{3}{r}{\small\slshape continued\ldots} \\
%	\endfoot

%%  The ending footer definition.                                   %%
%	\multicolumn{3}{c}{That's all folks} \\ \hline 
%	\endlastfoot
%%%%%%%%%%%%%%%%%%%%%%%%%%%%%%%%%%%%%%%%%%%%%%%%%%%%%%%%%%%%%%%%%%%%%%

\hhline{|-|-|-}
	 \multicolumn{1}{|p{\gnumericColA}|}%
	{\gnumericPB{\centering}\gnumbox{\textbf{Item}}}
	&\multicolumn{1}{p{\gnumericColB}|}%
	{\gnumericPB{\centering}\gnumbox{\textbf{Point}}}
	&\multicolumn{1}{p{\gnumericColC}|}%
	{\gnumericPB{\centering}\gnumbox{\textbf{Plane}}}
\\
\hhline{|---|}
	 \multicolumn{1}{|p{\gnumericColA}|}%
	{a)}
	&\multicolumn{1}{p{\gnumericColB}|}%
	{\gnumericPB{\centering}\gnumbox{\myvec{0\\0\\0}}}
	&\multicolumn{1}{p{\gnumericColC}|}%
	{\gnumericPB{\centering}\gnumbox{$\myvec{3 & -4 & 12}\bm{x}=3$}}
\\
\hhline{|---|}
	 \multicolumn{1}{|p{\gnumericColA}|}%
	{b)}
	&\multicolumn{1}{p{\gnumericColB}|}%
	{\gnumericPB{\centering}\gnumbox{\myvec{3\\-2\\1}}}
	&\multicolumn{1}{p{\gnumericColC}|}%
	{\gnumericPB{\centering}\gnumbox{$\myvec{2 & -1 & 2}\bm{x}=-3$}}
\\
\hhline{|---|}
	 \multicolumn{1}{|p{\gnumericColA}|}%
	{c)}
	&\multicolumn{1}{p{\gnumericColB}|}%
	{\gnumericPB{\centering}\gnumbox{\myvec{2\\3\\-5}}}
	&\multicolumn{1}{p{\gnumericColC}|}%
	{\gnumericPB{\centering}\gnumbox{$\myvec{1 & 2 & -2}\bm{x}=9$}}
\\
\hhline{|---|}
	 \multicolumn{1}{|p{\gnumericColA}|}%
	{d)}
	&\multicolumn{1}{p{\gnumericColB}|}%
	{\gnumericPB{\centering}\gnumbox{\myvec{-6\\0\\0}}}
	&\multicolumn{1}{p{\gnumericColC}|}%
	{\gnumericPB{\centering}\gnumbox{$\myvec{2 & -3 & 6}\bm{x}=2$}}
\\
\hhline{|-|-|-|}
\end{tabular}

\ifthenelse{\isundefined{\languageshorthands}}{}{\languageshorthands{\languagename}}
\gnumericTableEnd

\caption{}
\label{table:3d}
\end{table}
%
\item Show that the line joining the origin to the point \myvec{2\\1\\1} is perpendicular to the line determined by the points \myvec{3\\5\\-1}, \myvec{4\\3\\-1}.
\item If the coordinates of the points $\bm{A}, \bm{B}, \bm{C}, \bm{D}$ be \myvec{1\\2\\3}, \myvec{4\\5\\7}, \myvec{-4\\3\\-6}, \myvec{2\\9\\2}, then find the angle between the lines $AB$ and $CD$.  
%
\item If the lines 
\begin{align}
\frac{x-1}{-3} = \frac{y-2}{2k} &= \frac{z-3}{2}, 
\\
\frac{x-3}{3k} = \frac{y-1}{1} &= \frac{z-6}{-5} ,
\end{align}
find the value of $k$.
\item Find the  equation of the line passing through \myvec{1\\2\\3} and perpendicular to the plane %
\begin{align}
\myvec{1 & 2 & -5}\vec{x}&=-9
\end{align}
\item Find the shortest distance between the lines 
%
\begin{align}
\vec{x} = \myvec{6 \\ 2 \\ 2} + \lambda_1 \myvec{1 \\ -2 \\ 2}  \text{ and }
\\
\vec{x} = \myvec{-4 \\ 0 \\ -1} + \lambda_2 \myvec{3 \\ -2 \\ -2}  
\end{align}
%
\item Find the coordinates of the point where the line through \myvec{5\\1\\6} and \myvec{3\\4\\1} crosses the YZ-plane.
\item Find the coordinates of the point where the line through \myvec{5\\1\\6} and \myvec{3\\4\\1} crosses the ZX-plane.
\item Find the equation of the plane passing through the point \myvec{-1\\3\\2} and perpendicular to each of the planes 
\begin{align}
\myvec{1 & 2 & 3}\vec{x}&=5
\\
\myvec{3 & 3 & 1}\vec{x}&=0
\end{align}
\item If the points \myvec{1\\1\\p} and \myvec{-3\\0\\1} be equidistant from the plane 
\begin{align}
\myvec{3 & 4 & -12}\vec{x}&=-13,
\end{align}
%
then find the value of $p$.
\item Find the equation of the plane passing through the line of intersection of the planes 
\begin{align}
\myvec{1 & 1 & 1}\vec{x}&=1 \text{ and }
\\
\myvec{2 & 3 & -1}\vec{x}&=-4
\end{align}
%
and parallel to the x-axis.
\item If $\vec{O}$ be the origin and the coordinates of $\vec{P}$ be \myvec{1\\2\\3}, then find the equation of the plane passing through $\vec{P}$ and perpendicular to $OP$.
%
\item Find the equation of the plane which contains the line of intersection of the planes 
%
\begin{align}
\myvec{1 & 2 & 3}\vec{x}&=4 
\\
\myvec{2 & 1 & -1}\vec{x}&=-5
\end{align}
%
and which is perpendicular to the plane 
\begin{align}
\myvec{5 & 3 & -6}\vec{x}&=-8
\end{align}
%
\item Find the vector equation of the line passing through \myvec{1\\2\\3} and parallel to the planes 
%
\begin{align}
\myvec{1 & -1 & 2}\vec{x}&=5
\\
\myvec{3 & 1 & 1}\vec{x}&=6
\end{align}
%

\item The planes 
%
\begin{align}
\myvec{2 & -1 & 4}\vec{x}&=5
\\
\myvec{5 & -\frac{5}{2} & 10}\vec{x}&=6
\end{align}
%
are 
%
\begin{enumerate}[itemsep=2pt]
\begin{multicols}{2}
\item Perpendicular
\item Parallel
\item intersect y-axis
\item passes through $\myvec{0\\0\\\frac{5}{4}}$
\end{multicols}
\end{enumerate}
%
\item Find the maximum and minimum values, if any of
the following functions given by 
%
\begin{enumerate}
\item $f(x) = \abs{x+2}-1$
\item $f(x) = -\abs{x+1}+3$
\item $h(x) = x+1, x \in \brak{-1,1}$.
\end{enumerate}
%
\item Using integration find the area of region bounded by the triangle whose vertices are \myvec{1\\ 0}, \myvec{2\\ 2} and \myvec{3\\ 1}.
%
\item  Using integration find the area of region bounded by the triangle whose vertices are (– 1, 0), (1, 3) and (3, 2).
\item  Using integration find the area of the triangular region whose sides have the equations $\myvec{2 & -1 }\vec{x} = -1$, $\myvec{3 & -1 }\vec{x} = -1$ and x = 4.
%
\item Find the area of the region bounded by the line $\myvec{3 & -1}\vec{x} = -2$, the x-axis and the ordinates $x = -1, x = 1$.
\item Find the area bounded by the curve $\abs{x}+\abs{y} = 1$.
\item Using the method of integration find the area of $\triangle ABC$, whose vertices are $\vec{A} = \myvec{ 2\\0 }, \vec{B} = \myvec{ 4\\5 }, \vec{C} = \myvec{ 6\\3 }$.
\item  Using integration find the area of the triangular region whose sides have the equations $\myvec{2 & 1 }\vec{x} = 4$, $\myvec{3 & -2 }\vec{x} = 6$ and  $\myvec{1 & -3 }\vec{x} = -5$.
\item The two equal sides of an isosceles triangle with fixed base $b$ are decreasing at the rate of 3 cm per second. How fast is the area decreasing when the two equal sides are equal to the base ?
\item A tank with rectangular base and rectangular sides, open at the top is to be constructed so that its depth is 2 m and volume is 8 $m^3$
. If building of tank costs
\rupee 70 per sq metres for the base and Rs 45 per square metre for sides. What is the cost of least expensive tank?
\item A point on the hypotenuse of a triangle is at distance a and b from the sides of the triangle.
Show that the minimum length of the hypotenuse is
%
\begin{align}
\brak{a^{\frac{2}{3}}+b^{\frac{2}{3}}}^{\frac{3}{2}}
\end{align}
%
\item Prove that the function $f(x) = 5x – 3$ is continuous at $x = 0, at x = – 3$ and at $x = 5$.
\item Examine the following functions for continuity.
%
\begin{enumerate}
\item $f(x) = x-5$
\item $f(x) = \abs{x-1}$
\end{enumerate}
%
\item Is the function defined by 
%
\begin{align}
f(x)=
\begin{cases}
x, & x \le 1,
\\
5, & x > 1
\end{cases}
\end{align}
%
continuous at $x = 0$? At $x = 1$? At $x = 2$?
\item Find all points of discontinuity of $f$, where $f$ is defined by
%
\begin{enumerate}
\item 
$
\begin{alignedat}[t]{2}
f(x)=
\begin{cases}
2x+3, & x \le 2,
\\
2x-3, & x > 2
\end{cases}
\end{alignedat}
$
%
\item 
$
\begin{alignedat}[t]{2}
f(x)=
\begin{cases}
\abs{x}+3, & x \le -3,
\\
-2x, & -3 < x < 3
\\
6x+2, & x \ge 2
\end{cases}
\end{alignedat}
$
\item 
$
\begin{alignedat}[t]{2}
f(x)=
\begin{cases}
\frac{\abs{x}}{x}, & x \ne 0,
\\
0, & x = 0,
\end{cases}
\end{alignedat}
$
\item 
$
\begin{alignedat}[t]{2}
f(x)=
\begin{cases}
\frac{x}{\abs{x}}, & x < 0,
\\
-1, & x \ge 0,
\end{cases}
\end{alignedat}
$
\end{enumerate}
%
\item Is the function defined by 
%
\begin{align}
f(x)=
\begin{cases}
x+5, & x \le 1,
\\
x-5, & x > 1
\end{cases}
\end{align}
%
a continuous function?
\item Discuss the continuity of the function $f$, where $f$ is defined by 
\begin{enumerate}
\item 
$
\begin{alignedat}[t]{2}
f(x)=
\begin{cases}
3, & 0 \le x \le 1,
\\
4, & 0 < x \le 3,
\\
5, & 3 \le x \le 10,
\end{cases}
\end{alignedat}
$
%
\item 
$
\begin{alignedat}[t]{2}
f(x)=
\begin{cases}
2x, & x < 0,
\\
0, & 0 \le x \le 1
\\
4x, &  x > 1
\end{cases}
\end{alignedat}
$
\item 
$
\begin{alignedat}[t]{2}
f(x)=
\begin{cases}
-2, & x < -1,
\\
2x, & -1 \le x \le  1
\\
2, & x >  1
\end{cases}
\end{alignedat}
$
\end{enumerate}
%
\item Find the relationship between $a$ and $b$ so that the function defined by 	
%
\begin{align}
f(x)=
\begin{cases}
ax+1, & x \le 3,
\\
bx+3, & x > 3
\end{cases}
\end{align}
%
is continuous at $x = 3$
%
\item Prove that the function $f(x) = x$ is continuious at every real number.
\item Is $f(x) = \abs{x}$ a continuous function?
\item Discuss the continuity of the function $f$ defined by 
%
\begin{align}
f(x)  = 
\begin{cases}
x+2 & x \le 1
\\
x-2 & x > 1
\end{cases}
\end{align}

\item Show that the function defined by $g (x) = x – [x]$ is discontinuous at all integral points. Here $[x]$ denotes the greatest integer less than or equal to $x$.
\item For what value of $k$ is the following function 
%
continuous at the given point.
\begin{align}
f(x)=
\begin{cases}
kx+1, & x \le 5,
\\
3x-5, & x > 5,
\end{cases}
\quad x = 5
\end{align}
\item Prove that the function $f$ given by 
\begin{align}
f(x) = \abs{x-1}, x \in \vec{R}
\end{align}
%
is not differentiable at $x = 1$.
\item Prove that the greatest integer function defined by 
\begin{align}
f(x) = \abs{x}, 0 < x < 3
\end{align}
%
is not differentiable at $x = 1$ and $x = 2$.
\item Examine if Rolle's theorem is applicable to the following functions
\begin{enumerate}
\item 
\label{prob:line_eq_rolle}
$
f(x) = \sbrak{x}, x \in \sbrak{5,9}.
$
\item 
$
f(x) = \sbrak{x}, x \in \sbrak{-2,2}.
$
\end{enumerate}
Can you say some thing about the converse of Rolle's theorem from this example?
\item  Examine the applicability of the mean value theorem for all functions in Problem \ref{prob:line_eq_rolle}.
%
\item Find $\lim_{x\to 5} x+10$
\item Find $\lim_{x\to 2} 3x$
\item Find $\lim_{x\to 0}f(x)$ where
%
\begin{align}
f(x)  = 
\begin{cases}
1 & x \le 0
\\
2 & x > 0
\end{cases}
\end{align}
\item Find $\lim_{x\to 0}f(x)$ where
%
\begin{align}
f(x)  = 
\begin{cases}
x-2 & x < 0
\\
0 & x = 0
\\
x+2 & x > 0
\end{cases}
\end{align}
\item Evaluate the following limits
\begin{enumerate}
\item $\lim_{x\to 3}x+3$
\item $\lim_{x\to \pi}\brak{x-\frac{22}{7}}$
\end{enumerate}
%
\item Find $\lim_{x\to 0} f(x)$ where
\begin{align}
f(x) = 
\begin{cases}
\frac{\abs{x}}{x} & x \ne 0
\\
0, & x = 0
\end{cases}
\end{align}
%
\item Find $\lim_{x\to 0} f(x)$ where
\begin{align}
f(x) = 
\begin{cases}
\frac{x}{\abs{x}} & x \ne 0
\\
0, & x = 0
\end{cases}
\end{align}
%
\item Find $\lim_{x\to 5} \abs{x}-5$.
%
\item Suppose
\begin{align}
f(x) = 
\begin{cases}
a+bx & x \ne 1
\\
4, & x = 1
\\
b-ax & x > 1
\end{cases}
\end{align}
%
and if $\lim_{x\to 1}f(x) = f(1)$, what are the possible values of $a$ and $b$?
%
\item If
\begin{align}
f(x) = 
\begin{cases}
\abs{x}+1 & x < 0
\\
0, & x = 0
\\
\abs{x}-1 & x > 0
\end{cases}
\end{align}
%
for what value(s) of $a$ does $\lim_{x\to a}f(x)$ exists?

\item Find the derivative of $x$ at $x = 1$.
\item Find the derivative of $99x$ at $x = 100$.

\item Find the derivative of the following functions:
%
\begin{enumerate}
\item  $-x$
\item  $x+a$
\end{enumerate}
%
\item Integrate the following as limit of sums:
\begin{enumerate}[label = (\roman*)]
\item $\int_{a}^{b}x\, dx$
\item $\int_{0}^{5}\brak{x+1}\, dx$
\item $\int_{-1}^{1}\brak{x+1}\, dx$
\item $\int_{-5}^{5}\abs{x+2}\, dx$
\item $\int_{2}^{8}\abs{x-5}\, dx$
\item $\int_{0}^{4}\abs{x-1}\, dx$
\item $\int_{1}^{4}\sbrak{\abs{x-1}+\abs{x-2}+\abs{x-3}}\, dx$
\end{enumerate}
%
\item Form the differential equation representing the following family of curves 
\begin{align}
\myvec{\frac{1}{a} & \frac{1}{b}}\vec{x} = 1
\end{align}
%
\item Find $\theta$ and $p$ if 
%
\begin{align}
\myvec{\sqrt{3} & 1}\vec{x} = -2
\end{align}
%
is equivalent to
%
\begin{align}
\myvec{\cos\theta & \sin\theta}\vec{x} = p
\end{align}
\item Find the equation of the line which passes through  the point \myvec{-2\\4\\-5} and parallel to the line given by 
\begin{align}
\frac{x+3}{3} = \frac{y-4}{5} = \frac{z+8}{6}. 
\end{align}
\item Find the angle between the following pair of lines
\begin{enumerate}
\item 
\begin{align}
\frac{x-2}{2} = \frac{y-1}{5} &= \frac{z+3}{-3}, 
\\
\frac{x+2}{-1} = \frac{y-4}{8} &= \frac{z-5}{4} 
\end{align}
\item 
\begin{align}
\frac{x}{2} = \frac{y}{2} &= \frac{z}{1}, 
\\
\frac{x-5}{4} = \frac{y-2}{1} &= \frac{z-3}{8} 
\end{align}
\end{enumerate}
\item Find the equation of a plane which is at a distance of 7 units from the origin and normal to \myvec{3\\5\\-6}.
%
\item  For the following planes, find the coordinates of the foot of the perpendicular drawn from the origin
\begin{enumerate}[itemsep=2pt]
\begin{multicols}{2}
\item
$
\myvec{2 & 3 & 4}\vec{x}=12
$
\item
$
\myvec{3 & 4 & -6}\vec{x}=0
$
\item
$
\myvec{1 & 1 & 1}\vec{x}=1
$
\item
$
\myvec{0 & 5 &0}\vec{x}=-8
$
\end{multicols}
\end{enumerate}

%\end{enumerate}

\renewcommand{\theequation}{\theenumi}
\begin{enumerate}[label=\arabic*.,ref=\thesubsection.\theenumi]
\numberwithin{equation}{enumi}

\item Solve the following pair of linear equations
%
\begin{enumerate}[itemsep=2pt]
\begin{multicols}{2}
\item
\begin{align}
\begin{split}
\myvec{p & q }\vec{x}&=p-q
\\
\myvec{q & -p }\vec{x}&=p+q
\end{split}
\end{align}
\item
\begin{align}
\begin{split}
\myvec{a & b }\vec{x}&=c
\\
\myvec{b & a }\vec{x}&=1+c
\end{split}
\end{align}
\item
\begin{align}
\begin{split}
\myvec{\frac{1}{a} & -\frac{1}{b} }\vec{x}&=0
\\
\myvec{a & b }\vec{x}&=a^2+b^2
\end{split}
\end{align}
%
\end{multicols}
\end{enumerate}
%
\item Solve the following pair of equations
\begin{align}
\begin{split}
\myvec{a-b & a+b }\vec{x}&=a^2-2ab-b^2
\\
\myvec{a+b & a+b }\vec{x}&=a^2+b^2
\end{split}
\end{align}

\item In $\triangle ABC$, Show that the centroid 
\begin{align}
\vec{O} = \frac{\vec{A}+\vec{B}+\vec{C}}{3}
\end{align}
%
\item (Cauchy-Schwarz Inequality:) Show that 
%
\begin{align}
\abs{\vec{a}^T\vec{b}} \le \norm{\vec{a}}\norm{\vec{b}}
\end{align}
%
%
\item (Triangle Inequality:) Show that 
%
\begin{align}
\norm{\vec{a}+\vec{b}} \le \norm{\vec{a}}+\norm{\vec{b}}
\end{align}
%
%
\item The base of an equilateral triangle with side $2a$ lies along the y-axis such that the mid-point of the base is at the origin. Find vertices of the triangle.
\item Find the distance between $\vec{P}= \myvec{x_1 y_1}$ and $\vec{Q} =\myvec{x_2 y_2}$ when
\begin{enumerate}
\item PQ is parallel to the y-axis.
\item PQ is parallel to the x-axis.
\end{enumerate}
\item If three points \myvec{h 0}, \myvec{a b} and \myvec{0 k} lie on a line, show that
$\frac{a}{h}+\frac{b}{k}= 1$.
\item $\vec{P}=\myvec{a b}$ is the mid-point of a line segment between axes. Show that equation of the line is
\begin{align}
\myvec{\frac{1}{a} & \frac{1}{b}}\vec{x} = 2
\end{align}
\item  Point $\vec{R}= \myvec{h k}$ divides a line segment between the axes in the ratio 1: 2. Find equation of the line.
\item Show that two lines 
\begin{align}
\myvec{a_1 & b_1}\vec{x} +c_1&= 0
\\
\myvec{a_2 & b_2}\vec{x} +c_2&= 0
\end{align}
are 
\begin{enumerate}
\item parallel if $\frac{a_1}{b_1}=\frac{a_2}{b_2}$ and
\item perpendicular if $a_1a_2-b_1b_2 = 0$.
\end{enumerate}
%
\item Find the distance between the parallel lines
%
\begin{align}
l\myvec{1 & 1}\vec{x} = -p
\\
l\myvec{1 & 1}\vec{x} = r
\end{align}
%
\item Find th equation of the line through the point $\vec{x}_1$ and parallel to the line
%
\begin{align}
\myvec{A & B}\vec{x} = -C
\end{align}
%
\item If $p$ and $q$ are the lengths of perpendiculars from the origin to the lines 
%
\begin{align}
\myvec{\cos\theta & \sin\theta}\vec{x} &= k\cos2\theta
\\
\myvec{\sec\theta & \cosec\theta}\vec{x} &= k
\end{align}
%
respectively, prove that $p^2+4q^2=k^2$.
\item If $p$ is the length of the perpendicular from the origin to the line whose intercepts on the axes are $a$ and $b$, then show that 
%
\begin{align}
\frac{1}{p^2} = \frac{1}{a^2}+\frac{1}{b^2}.
\end{align}
%
\item Show that the area of the triangle formed by the lines
%
\begin{align}
\myvec{-m_1 & 1}\vec{x} = c_1
\\
\myvec{-m_2 & 1}\vec{x} = c_2
\\
\myvec{1 & 0}\vec{x} = 0
\end{align}
%
is $\frac{\brak{c_1-c_2}^2}{2\abs{m_1-m_2}}$.
\item Find the values of $k$ for which the line 
%
\begin{align}
\myvec{k-3 & -\brak{4-k^2}}\vec{x} +k^2-7k+6= 0
\end{align}
%
is
\begin{enumerate}
\item parallel to the x-axis
\item parallel to the y-axis
\item passing through the origin.
\end{enumerate}
%
\item Find the perpendicular distance from the origin to the line joining the points \myvec{\cos\theta\sin\theta} and \myvec{\cos\phi \sin \phi}.
\item Find the area of the triangle formed by the lines
%
\begin{align}
\myvec{1 & -1}\vec{x} &= 0
\\
\myvec{1 & 1}\vec{x} &= 0
\\
\myvec{1 & 0}\vec{x} &= k
\end{align}
%
\item If three lines whose equations are 
%
\begin{align}
\myvec{-m_1 & 1}\vec{x} &= c_1
\\
\myvec{-m_2 & 1}\vec{x} &= c_2
\\
\myvec{-m_3 & 1}\vec{x} &= c_3
\end{align}
%
are concurrent, show that
%
\begin{align}
m_1\brak{c_2-c_3}+
m_2\brak{c_3-c_1}+
m_3\brak{c_1-c_2} = 0
\end{align}
%
\item Find the equation of the line passing through the origin and making an angle $\theta$ with the line %
\begin{align}
\myvec{-m & 1}\vec{x} &= c
\end{align}
%
\item Prove that the product of the lengths of the perpendiculars drawn from the points $\myvec{\sqrt{a^2-b^2}0}$ and $\myvec{\sqrt{a^2-b^2}0}$ to the line 
%
\begin{align}
\myvec{\frac{\cos\theta}{a} & \frac{\sin\theta}{b}}\vec{x} &= 1
\end{align}
%
is $b^2$.

\item If 
$
\myvec{l_1m_1n_1}
$
and
$
\myvec{l_2m_2n_2}
$
are the unit direction vectors of two mutually perpendicular lines, the shown that the unit direction vector of the line perpendicular to both of these is
$
\myvec{m_1n_2-m_2n_1n_1l_2-n_2l_1l_1m_2-l_2m_1}.
$
\item A line makes angles $\alpha, \beta, \gamma, \delta$ with the diagonals of a cube, prove that \begin{align}
\cos^2\alpha + \cos^2\beta + \cos^2\gamma +\cos^2\delta = \frac{4}{3}.
\end{align}
\item Show that the lines 
\begin{align}
\frac{x-a+d}{\alpha-\delta} = \frac{y-a}{\alpha} &= \frac{z-a-d}{\alpha+\delta}, 
\\
\frac{x-b+c}{\beta-\gamma} = \frac{y-b}{\beta} &= \frac{z-b-c}{\beta+\gamma} 
\end{align}
%
are coplanar.
\item Find $\vec{R}$ which divides the line joining the points 
\begin{align}
\vec{P} = 2\vec{a}+\vec{b}
\\
\vec{Q} = \vec{a}-\vec{b}
\end{align}
externally in the ratio $1:2$.
\item Find $\norm{\vec{a}}$ and $\norm{\vec{b}}$ if 
\begin{align}
\brak{\vec{a}+\vec{b}}^T\brak{\vec{a}-\vec{b}} &= 8
\\
\norm{\vec{a}}&=8\norm{\vec{b}}
\end{align}
\item Evaluate the product 
\begin{align}
\brak{3\vec{a}-5\vec{b}}^T\brak{2\vec{a}+7\vec{b}} 
\end{align}
\item Find $\norm{\vec{a}}$ and $\norm{\vec{b}}$, if
\begin{align}
\norm{\vec{a}} &= \norm{\vec{b}},
\\
\vec{a}^T\vec{b} = \frac{1}{2} 
\end{align}
and the angle between $\vec{a}$ and $\vec{b}$ is $60\degree$.
\item Show that 
\begin{align}
\brak{\norm{\vec{a}}\vec{b}+\norm{\vec{b}}\vec{a}}\perp \brak{\norm{\vec{a}}\vec{b}-\norm{\vec{b}}\vec{a}}\\
\end{align}
\item If $\vec{a}^T\vec{a}=0$ and  $\vec{a}\vec{b}=0$, what can be concluded about the vector $\vec{b}$?
\item If $\vec{a},\vec{b},\vec{c}$ are unit vectors such that 
\begin{align}
\vec{a}+\vec{b}+\vec{c} = 0,
\end{align}
find the value of 
\begin{align}
\vec{a}^T\vec{b}+\vec{b}^T\vec{c}+\vec{c}^T\vec{a}.
\end{align}
\item If $\vec{a} \ne \vec{0}, \lambda \ne 0$, then $\norm{\lambda \vec{a}} = 1$ if
\begin{enumerate}
\item $\lambda =1$
\item $\lambda = -1$
\item $\norm{\vec{a}}=\abs{\lambda}$
\item $\norm{\vec{a}}=\frac{1}{\abs{\lambda}}$
\end{enumerate}
\item If a unit vector $\vec{a}$ makes angles $\frac{\pi}{3}$ with the x-axis and $\frac{\pi}{4}$ with the y-axis and an acute angle $\theta$ with the z-axis, find $\theta$ and $\vec{a}$.
\item Show that 
\begin{align}
\brak{\vec{a}-\vec{b}}\times \brak{\vec{a}+\vec{b}} = 2\brak{\vec{a}\times\vec{b}}
\end{align}
\item If $\vec{a}^T\vec{b} = 0$ and $\vec{a}\times \vec{b}$ = 0, what can you conclude about $\vec{a}$ and $\vec{b}$?
\item Find $\vec{x}$ if  $\vec{a}$ is a unit vector such that
\begin{align}
\brak{\vec{x}-\vec{a}}^T\brak{\vec{x}+\vec{a}} = 12.
\end{align}
\item If $\norm{\vec{a}} = 3, \norm{\vec{b}} =\frac{\sqrt{2}}{3}$, then $\vec{a}\times \vec{b}$ is a unit vector if the angle between $\vec{a}$ and $\vec{b}$ is 
\begin{enumerate}[itemsep = 2pt]
\begin{multicols}{2}
\item $\frac{\pi}{6}$
\item $\frac{\pi}{4}$
\item $\frac{\pi}{3}$
\item $\frac{\pi}{2}$
\end{multicols}
\end{enumerate}
\item Prove that 
\begin{align}
\brak{\vec{a}+\vec{b}}^T\brak{\vec{a}+\vec{b}} &= \norm{\vec{a}}^2+\norm{\vec{b}}^2
\\
\iff \vec{a}&\perp\vec{b}.
\end{align}
\item If $\theta$ is the angle between two vectors $\vec{a}$ and $\vec{b}$, then $\vec{a}^T\vec{b} \ge $ only when 
\begin{enumerate}[itemsep = 2pt]
\begin{multicols}{2}
\item $0 < \theta < \frac{\pi}{2}$
\item $0 \le \theta \le \frac{\pi}{2}$
\item $0 < \theta < {\pi}$
\item $0 \le \theta \le {\pi}$
\end{multicols}
\end{enumerate}
\item Let $\vec{a}$ and $\vec{b}$ be two unit vectors and $\theta$ be the angle between them.  Then $\vec{a}+\vec{b}$ is a unit vector if 
\begin{enumerate}[itemsep = 2pt]
\begin{multicols}{2}
\item $\theta = \frac{\pi}{4}$
\item $\theta = \frac{\pi}{3}$
\item $\theta = \frac{\pi}{2}$
\item $\theta = \frac{2\pi}{3}$
\end{multicols}
\end{enumerate}
\item If $\theta$ is the angle between any two vectors $\vec{a}$ and $\vec{b}$, then 
$\norm{\vec{a}^T\vec{b}} = \norm{\vec{a} \times \vec{b}}$ when $\theta$ is equal to 
\begin{enumerate}[itemsep = 2pt]
\begin{multicols}{2}
\item 0
\item $\frac{\pi}{4}$
\item $\frac{\pi}{2}$
\item $\pi$.
\end{multicols}
\end{enumerate}
\item Find the angle between the lines whose direction vectors are $\myvec{abc}$ and $\myvec{b-cc-aa-b}$.
\item Find the equation of a line parallel to the x-axis and passing through the origin.
\item Find the equation of a plane passing through \myvec{abc} and parallel to the plane 
%
\begin{align}
\myvec{1 & 1 & 1}\vec{x}{x}&=2
\end{align}
%
\item Prove that if a plane has the intercepts $a, b, c$ and is at a distance of $p$ units from the origin, then, 
\begin{align}
\frac{1}{a^2}+\frac{1}{b^2}+\frac{1}{c^2}=\frac{1}{p^2} 
\end{align}
     \item In an experiment, a solution of hydrochloric acid is to be kept between 30$\degree$ and 35$\degree$ Celsius. What is the range of temperature in degree Fahrenheit if conversion formula is given by 
     C = $\frac{5}{9}(F-32)$, where C and F represent temperature in degree Celsius and degree Fahrenheit, respectively.
     \item A manufacturer has 600 litres of a 12$\%$ solution of acid. How many litres of a 30$\%$ acid solution must be added to it so that acid content in the resulting mixture will be more than 15$\%$ but less than 18$\%$?
    \item Ravi obtained 70 and 75 marks in first two unit test. Find the minimum marks he should get in the third test to have an average of at least 60 marks.
    \item To receive Grade A in a course, one must obtain an average of 90 marks or more in five examinations (each of 100 marks). If Sunita’s marks in first four examinations are 87, 92, 94 and 95, find minimum marks that Sunita must obtain in fifth examination to get grade ‘A’ in the course.
    \item Find all pairs of consecutive odd positive integers both of which are smaller than 10 such that their sum is more than 11.
    \item Find all pairs of consecutive even positive integers, both of which are larger than 5 such that their sum is less than 23.
    \item A man wants to cut three lengths from a single piece of board of length 91cm.The second length is to be 3cm longer than the shortest and the third length is to be twice as long as the shortest. What are the possible lengths of the shortest board if the third piece is to be at least 5cm longer than the second?
    \item A solution is to be kept between 68$\degree$F and 77$\degree$F. What is the range in temperature in degree Celsius(C) if the Celsius / Fahrenheit(F) conversion formula is given by
    F= $\frac{9}{5}$ C+32?
    \item A solution of 8$\%$ boric acid is to be diluted by adding a 2$\%$ boric acid solution to
it. The resulting mixture is to be more than 4$\%$ but less than 6$\%$ boric acid. If we have
640 litres of the 8$\%$ solution, how many litres of the 2$\%$ solution will have to be added?
    \item How many litres of water will have to be added to 1125 litres of the 45$\%$ solution
of acid so that the resulting mixture will contain more than 25$\%$ but less than 30$\%$ acid
content?
    \item IQ of a person is given by the formula
            IQ = $\frac{MA}{CA}\times 100$,
    where MA is mental age and CA is chronological age.If 80 $\leq$ IQ $\leq$ 140 for a group of
12 years old children, find the range of their mental age.
%
\end{enumerate}

\end{enumerate}


\end{document}


