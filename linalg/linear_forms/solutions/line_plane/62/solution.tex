
Given that
\begin{align}
\vec{A}=\myvec{1 \\0}   and \ \vec{B}= \myvec{2 \\3}
\end{align}

The line RP intersect the line AB in 1:n ration, using section formula


\begin{align}
\vec{P}=\frac{\vec{B}+n\vec{A}}{n+1}  
\end{align}
Using equations (1.1.1) and (1.1.3), 

\begin{align}
\vec{P} = \myvec{\frac{n+2}{n+1}\\\frac{3}{n+1}}
\end{align}

Direction vector of line AB
\begin{align}
 \vec{m}=\myvec{2\\3}-\myvec{1\\0} = \myvec{1\\3} 
\end{align}

Let $\vec{x}$ is the point on line PR, direction vector of line PR will be \myvec{\vec{x}-\Vec{P}}

Since line AB and line PR are perpendicular to each other, dot product of direction vectors will be zero.

Therefore, 
\begin{align}
\myvec{\vec{m}}^T\myvec{\vec{x}-\vec{P}}=0
\end{align}




\begin{align}
\vec{m}^T\vec{x}=\vec{m}^T\vec{P}
\end{align}



Putting the values of $\vec{m}$, $\vec{x}$ and $\vec{p} $ in equation (1.1.6)
\begin{align}
\myvec{1&3}\vec{x}=\myvec{1& 3}\myvec{\frac{n+2}{n+1}\\\frac{3}{n+1}}
\end{align}

Solving the equation (1.1.7), 
equation of the line PR is
\begin{align}
\myvec{1&3}\vec{x} = \frac{n+11}{n+1}
\end{align}
\end{document}
