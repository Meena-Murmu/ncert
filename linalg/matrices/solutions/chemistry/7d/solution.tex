We know that the number of atoms of each element remains the
same, before and after a chemical reaction.

Equation \eqref{eq:solutions/chemistry/7d:1} can be written as:
\begin{align}
x_1BaCl_2 + x_2K_2SO_4 \rightarrow x_3BaSO_4 + x_4KCl\label{eq:solutions/chemistry/7d:2}
\end{align}
Element wise contribution in forming the respective chemical compound can be written in the form of equation as :
\begin{align}
Ba : x_1 + 0x_2 - x_3 - 0x_4 = 0\\
Cl : 2x_1 + 0x_2 - 0x_3 - 1x_4 = 0\\
K  : 0x_1 + 2x_2 - 0x_3 - 1x_4 = 0\\
S  : 0x_1 + 1x_2 - 1x_3 - 0x_4 = 0\\
O  : 0x_1 + 4x_2 - 4x_3 - 0x_4 = 0
\end{align}
In matrix form this can be written as:
\begin{align}
A\vec{x}&=0\\
  \myvec{1&0&-1&0\\2&0&0&-1\\0&2&0&-1\\0&1&-1&0\\0&4&-4&0}\myvec{x_1\\x_2\\x_3\\x_4}&=\myvec{0\\0\\0\\0\\0}
\end{align}
Using Gaussian Elimination method 
\begin{align}
\xleftrightarrow{R_2 \leftrightarrow R_5}\myvec{1&0&-1&0&:&0\\0&4&-4&0&:&0\\0&2&0&-1&:&0\\0&1&-1&0&:&0\\2&0&0&-1&:&0}\\
\xleftrightarrow{R_5 \leftarrow 2R_1-R_5}\myvec{1&0&-1&0&:&0\\0&4&-4&0&:&0\\0&2&0&-1&:&0\\0&1&-1&0&:&0\\0&0&-2&1&:&0}\\
\xleftrightarrow[R_4 \leftarrow 4R_4-R_2]{R_3 \leftarrow 2R_3-R_2}\myvec{1&0&-1&0&:&0\\0&4&-4&0&:&0\\0&0&4&-2&:&0\\0&0&0&0&:&0\\0&0&-2&1&:&0}\\
\xleftrightarrow{R_5 \leftrightarrow R_5}\myvec{1&0&-1&0&:&0\\0&4&-4&0&:&0\\0&0&4&-2&:&0\\0&0&-2&1&:&0\\0&0&0&0&:&0}\\
\xleftrightarrow{R_4 \leftarrow 2R_4-R_3}\myvec{1&0&-1&0&:&0\\0&4&-4&0&:&0\\0&0&4&-2&:&0\\0&0&0&0&:&0\\0&0&0&0&:&0}
\end{align}
Clearly the system is linearly dependent. Therefore by fixing the value of $x_4=2$, one of the possible vector $\vec{x}$ is:
\begin{align}
    \vec{x}=\myvec{1\\1\\1\\2}
\end{align}
Hence by putting the values of $x_1,x_2,x_3,x_4$ in equation \eqref{eq:solutions/chemistry/7d:1} we get our balanced chemical equation as followes :
\begin{align}
BaCl_2 + K_2SO_4 \rightarrow BaSO_4 + 2KCl
\end{align}


