	The given equations can be written as
\begin{align*}
	\vec{A} \vec{x} = \vec{b}  
\end{align*}
where
\begin{align}
	\vec{A} = \myvec{ 5 & -1 & 4 \\ 2 &  3 & 5 \\ 5 & -2 & 6 }
	\quad \text{and} \quad
	\vec{b} = \myvec{ 5 \\  2 \\ -1}
\end{align}
By row reducing the augmented matrix :
\begin{align}
	\myvec{ 5 & -1 & 4 &\vrule & 5 \\ 2 & 3 & 5 &\vrule & 2 \\ 5 & -2 & 6 &\vrule & -1}
	\\
	\xleftrightarrow[R_3 \leftarrow R_3-R_1]{R_2 \leftarrow 5R_2-(R_1+R_3)}
	\myvec{5 &-1  & 4  &\vrule & 5 \\ 0 & 18 & 15 &\vrule & 6 \\ 0 & -1 & 2 &\vrule & -6}
	\\
	\xleftrightarrow[]{R_3 \leftarrow 18R_3+R_2}
	\myvec{5 & -1 &  4 &\vrule & 5 \\ 0 & 18 & 15 &\vrule & 6 \\ 0 & 0 & 51 &\vrule & -102}
	\\
	\xleftrightarrow[]{R_3 \leftarrow \frac{R_3}{51}}
	\myvec{5 & -1 &  4 &\vrule & 5 \\ 0 & 18 & 15 &\vrule & 6 \\ 0 & 0 & 1 &\vrule & -2}
	\\
	\xleftrightarrow[]{R_2 \leftarrow R_2-15 R_3}
	\myvec{5 & -1 &  4 &\vrule & 5 \\ 0 & 18 & 0 &\vrule & 36 \\ 0 & 0 & 1 &\vrule & -2}
	\\
	\xleftrightarrow[]{R_2 \leftarrow \frac{R_2}{18}}
	\myvec{5 & -1 &  4 &\vrule & 5 \\ 0 & 1 & 0 &\vrule & 2 \\ 0 & 0 & 1 &\vrule & -2}
\end{align}
%----------------
\begin{align}
	\xleftrightarrow[]{R_1 \leftarrow R_1+R_2-4 R_3}
	\myvec{5 & 0 &  0 &\vrule & 15 \\ 0 & 1 & 0 &\vrule & 2 \\ 0 & 0 & 1 &\vrule & -2}
	\\
	\xleftrightarrow[]{R_1 \leftarrow \frac{R_1}{5}}
	\myvec{1 & 0 &  0 &\vrule & 3 \\ 0 & 1 & 0 &\vrule & 2 \\ 0 & 0 & 1 &\vrule & -2}
\end{align}
%----------------
\begin{align}
	\implies rank \myvec{ 5 & -1 & 4 \\ 2 &  3 & 5 \\ 5 & -2 & 6 }
		&=
	rank \myvec{ 5 & -1 & 4 &\vrule & 5 \\ 2 & 3 & 5 &\vrule & 2 \\ 5 & -2 & 6 &\vrule & -1}
		\nonumber \\
	&= 3 = dim \myvec{ 5 & -1 & 4 \\ 2 &  3 & 5 \\ 5 & -2 & 6 }
\end{align}
i.e., the $rank(\vec{A}) = rank(\vec{A:b}) = 3$, which is equal to the row size of $\vec{x}$. Hence the system of linear equations is consistent, with a unique solution. 

The unique solution is 
\begin{align}
	\vec{x} = \myvec{3\\2\\-2}
\end{align}
