Characteristic equation of A is $\lambda^2 - 2\lambda + 1 = 0$.
By using Cayley Hamolton theorem, given matrix A is a solution of its own characteristic equation. So $A^2 - 2A + I = 0$.
\begin {align}
	A^2 &= 2A - I  \\
	A^3 &= 3A - 2I \\
        A^4 &= 4A - 3I \\
	A^5 &= 5A - 4I
\end{align}
 By the substitution, 
\begin {align}
	A^n &= nA - (n-1)I 
\end{align}
\begin{enumerate}
\item Proving above equation is true for n=2,
\begin {align}
A^2 = 2A -I
\end {align}

\item Assume it is true for n = k
 \begin {align}
A^k = kA -(k-1)I
\end {align}

\item Proving it is true for n = k+1
 \begin {align}
A^{k+1} &= A^k A \\
        &= (kA -(k-1)I)A \\  
        &= kA^2 - (k-1)A \\
        &= k(2A -I) - (k-1)A \\
        &= (2k-k+1)A - kI \\
        &= (k+1)A - kI 
\end {align}
\end{enumerate}
 So, $A^n = nA - (n-1)I$ 

Substituting matrix A = \myvec{ 3 & -4 \\ 1 & -1 } in the equation5,
\begin {align}
	A^n &= n\myvec{ 3 & -4 \\ 1 & -1 } - (n-1)\myvec{ 1 & 0 \\ 0 & 1 } \\
	    &=\myvec{ 3n & -4n \\ n & -n } - \myvec{ (n-1) & 0 \\ 0 & (n-1)} \\
	    &=\myvec{ 2n+1 & -4n \\ n & 1-2n }  
\end{align}
So, $A^n =\myvec{ 2n+1 & -4n \\ n & 1-2n }$  

