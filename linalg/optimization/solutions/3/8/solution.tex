
Find the distance of the point \myvec{2\\3\\-5} from the plane \myvec{1&2&-2}$\vec{x}$ = 9
First we find orthogonal vectors $\vec{m_1}$ and $\vec{m_2}$ to the given normal vector $\vec{n}$. Let, $\vec{m}$ = $\myvec{a\\b\\c}$, then
\begin{align}
\vec{m^T}\vec{n} &= 0\\
\implies\myvec{a&b&c}\myvec{1\\2\\-2} &= 0\\
\implies a+2b-2c &= 0
\end{align}
Putting a=1 and b=0 we get,
\begin{align}
\vec{m_1} &= \myvec{1\\0\\\frac{1}{2}}\end{align}
Putting a=0 and b=1 we get,
\begin{align}
\vec{m_2} &= \myvec{0\\1\\1}
\end{align}
Now we solve the equation,
\begin{align}
\vec{M}\vec{x} &= \vec{b}\label{eq:solutions/3/8/eq1}\\
\intertext{Putting values in \eqref{eq:solutions/3/8/eq1},}
\myvec{1&0\\0&1\\\frac{1}{2}&1}\vec{x} &= \myvec{2\\3\\-5}\label{eq:solutions/3/8/eq2}
\end{align}
In order to solve \eqref{eq:solutions/3/8/eq2},  perform Singular Value Decomposition on $\vec{M}$ as follows,
\begin{align}
\vec{M}=\vec{U}\vec{S}\vec{V}^T\label{eq:solutions/3/8/eq100}
\end{align}
Where the columns of $\vec{V}$ are the eigen vectors of $\vec{M}^T\vec{M}$ ,the columns of $\vec{U}$ are the eigen vectors of $\vec{M}\vec{M}^T$ and $\vec{S}$ is diagonal matrix of singular value of eigenvalues of $\vec{M}^T\vec{M}$.
\begin{align}
\vec{M}^T\vec{M}=\myvec{\frac{5}{4}&\frac{1}{2}\\\frac{1}{2}&2}\label{eq:solutions/3/8/eqMTM}\\
\vec{M}\vec{M}^T=\myvec{1&0&\frac{1}{2}\\0&1&1\\\frac{1}{2}&1&\frac{5}{4}}
\end{align}
From \eqref{eq:solutions/3/8/eq1} putting \eqref{eq:solutions/3/8/eq100} we get,
\begin{align}\label{eq:solutions/3/8/eqX}
\vec{U}\vec{S}\vec{V}^T\vec{x} & = \vec{b}\\
\implies\vec{x} &= \vec{V}\vec{S_+}\vec{U^T}\vec{b}
\end{align}
Where $\vec{S_+}$ is Moore-Penrose Pseudo-Inverse of $\vec{S}$.Now, calculating eigen value of $\vec{M}\vec{M}^T$,
\begin{align}
\mydet{\vec{M}\vec{M}^T - \lambda\vec{I}} &= 0\\
\implies\myvec{1-\lambda&0&\frac{1}{2}\\0&1-\lambda&1\\\frac{1}{2}&1&\frac{5}{2}-\lambda} &=0\\
\implies-4\lambda^3+13 \lambda^2-9\lambda &=0
\end{align}
Hence eigen values of $\vec{M}\vec{M}^T$ are,
\begin{align}
\lambda_1 &=\frac{9}{4}\\
\lambda_2 &= 1\\
\lambda_3 &=0
\end{align}
Hence the eigen vectors of $\vec{M}\vec{M}^T$ are,
\begin{align}
\vec{u}_1=\myvec{\frac{2}{5}\\\frac{4}{5}\\1},
\vec{u}_2=\myvec{-2\\1\\0},
\vec{u}_3=\myvec{\frac{-1}{2}\\-1\\1}
\end{align}
Normalizing the eigen vectors we get,
\begin{align}
\vec{u}_1=\myvec{\frac{2}{\sqrt{45}}\\\frac{4}{\sqrt{45}}\\\frac{5}{\sqrt{45}}},
\vec{u}_2=\myvec{-\frac{2}{\sqrt{5}}\\\frac{1}{\sqrt{5}}\\0},
\vec{u}_3=\myvec{-\frac{1}{3}\\-\frac{2}{3}\\\frac{2}{3}}
\end{align}
Hence we obtain $\vec{U}$ of \eqref{eq:solutions/3/8/eq100} as follows,
\begin{align}\label{eq:solutions/3/8/eqU}
\myvec{\frac{2}{\sqrt{45}}& -\frac{2}{\sqrt{5}}&-\frac{1}{3}\\
\frac{4}{\sqrt{45}}&\frac{1}{\sqrt{5}}&-\frac{2}{3}\\
\frac{5}{\sqrt{45}}&0&\frac{2}{3}}
\end{align}
After computing the singular values from eigen values $\lambda_1, \lambda_2, \lambda_3$ we get $\vec{S}$ of \eqref{eq:solutions/3/8/eq100} as follows,
\begin{align}\label{eq:solutions/3/8/eqS}
\vec{S}=\myvec{\frac{3}{2}&0\\0&1\\0&0}
\end{align}
Now, calculating eigen value of $\vec{M}^T\vec{M}$,
\begin{align}
\mydet{\vec{M}^T\vec{M} - \lambda\vec{I}} &= 0\\
\implies\myvec{\frac{5}{4}-\lambda&\frac{1}{2}\\\frac{1}{2}&2-\lambda} &=0\\
\implies\lambda^2-\frac{13}{4}\lambda+\frac{9}{4} &=0
\end{align}
Hence eigen values of $\vec{M}^T\vec{M}$ are,
\begin{align}
\lambda_4 &= \frac{9}{4}\\
\lambda_5 &=1
\end{align}
Hence the eigen vectors of $\vec{M}^T\vec{M}$ are,
\begin{align}
\vec{v}_1=\myvec{\frac{1}{2}\\1},
\vec{v}_2=\myvec{-2\\1}
\intertext{Normalizing the eigen vectors we get,}
\vec{v}_1=\myvec{\frac{1}{\sqrt{5}}\\\frac{2}{\sqrt{5}}},
\vec{v}_2=\myvec{-\frac{2}{\sqrt{5}}\\\frac{1}{\sqrt{5}}}
\end{align}
Hence we obtain $\vec{V}$ of \eqref{eq:solutions/3/8/eq100} as follows,
\begin{align}
\vec{V}=\myvec{\frac{1}{\sqrt{5}}&-\frac{2}{\sqrt{5}}\\\frac{2}{\sqrt{5}}&\frac{1}{\sqrt{5}}}
\end{align}
 From \eqref{eq:solutions/3/8/eq100} we get the Singular Value Decomposition of $\vec{M}$ ,
\begin{align}
\vec{M} = \myvec{\frac{2}{\sqrt{45}}&-\frac{2}{\sqrt{5}}&-\frac{1}{3}\\
\frac{4}{\sqrt{45}} & \frac{1}{\sqrt{5}}&\frac{-2}{3}\\
\frac{5}{\sqrt{45}}  & 0&\frac{2}{3}}\myvec{\frac{3}{2}&0\\0&1\\0&0}\myvec{\frac{1}{\sqrt{5}}&-\frac{2}{\sqrt{5}}\\\frac{2}{\sqrt{5}}&\frac{1}{\sqrt{5}}}^T
\end{align}
Moore-Penrose Pseudo inverse of $\vec{S}$ is given by,
\begin{align}
\vec{S_+} = \myvec{\frac{2}{3}&0&0\\0&1&0}
\end{align}
From \eqref{eq:solutions/3/8/eqX} we get,
\begin{align}
\vec{U}^T\vec{b}&=\myvec{-\frac{3\sqrt{5}}{5}\\-\frac{\sqrt{5}}{5}\\-6}\\
\vec{S_+}\vec{U}^T\vec{b}&=\myvec{-\frac{2\sqrt{5}}{5}\\-\frac{\sqrt{5}}{5}}\\
\vec{x} = \vec{V}\vec{S_+}\vec{U}^T\vec{b} &= \myvec{0\\-1}\label{eq:solutions/3/8/eq85}
\end{align}
Verifying the solution of \eqref{eq:solutions/3/8/eq85} using,
\begin{align}
\vec{M}^T\vec{M}\vec{x} = \vec{M}^T\vec{b}\label{eq:solutions/3/8/eqVerify}
\end{align}
Evaluating the R.H.S in \eqref{eq:solutions/3/8/eqVerify} we get,
\begin{align}
\vec{M}^T\vec{M}\vec{x} &= \myvec{-\frac{1}{2}\\-2}\\
\implies\myvec{\frac{5}{4}&\frac{1}{2}\\\frac{1}{2}&2}\vec{x} &= \myvec{-\frac{1}{2}\\{-2}}\label{eq:solutions/3/8/eq:eq17}
\end{align}
Solving the augmented matrix of \eqref{eq:solutions/3/8/eq:eq17} we get,
\begin{align}
\myvec{\frac{5}{4}&\frac{1}{2}&-\frac{1}{2}\\\frac{1}{2}&2&-2} &\xleftrightarrow{R_1=\frac{4}{5}R_1}\myvec{1&\frac{2}{5}&\frac{-2}{5}\\\frac{1}{2}&2&-2}\\
&\xleftrightarrow{R_2=R_2-\frac{1}{2}R_2}\myvec{1&-\frac{2}{5}&-\frac{2}{5}\\0&\frac{9}{5}&-\frac{9}{5}}\\
&\xleftrightarrow{R_2=\frac{5}{9}R_2}\myvec{1&\frac{2}{5}&-\frac{2}{5}\\0&1&-1}\\
&\xleftrightarrow{R_1=R_1-\frac{2}{5}R_2}\myvec{1&0&0\\0&1&-1}\label{eq:solutions/3/8/eq:eq13}
\end{align}
From equation \eqref{eq:solutions/3/8/eq:eq13}, solution is given by,
\begin{align}\label{eq:solutions/3/8/eq:eq14}
\vec{x}=\myvec{0\\-1}
\end{align}
Comparing results of $\vec{x}$ from \eqref{eq:solutions/3/8/eq85} and \eqref{eq:solutions/3/8/eq:eq14}, we can say that the solution is verified.

 

   
