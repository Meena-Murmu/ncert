From the given information	
	\begin{equation}\label{eq:solution_point_vector_15_1}
		\vec{d}^{T}\vec{a} = 0
	\end{equation} 		

	Similarly, as $\vec{d}\perp\vec{b}$
 
	\begin{equation}\label{eq:solution_point_vector_15_3}
		\vec{d}^{T}\vec{b} = 0 
	\end{equation}


	It is given that
	
	\begin{equation}\label{eq:solution_point_vector_15_5}
		\vec{d}^{T}\vec{c} = 15 
	\end{equation}

	
    Using equations \ref{eq:solution_point_vector_15_1}, \ref{eq:solution_point_vector_15_3}, \ref{eq:solution_point_vector_15_5}, we can represent them in a Matrix Representation of Linear Equations $A$$x$=$B$ form as:
 
    	\begin{align} \label{eq:solution_point_vector_15_6}
    		\myvec{
    			\vec{a}^{T} \\
    			\vec{b}^{T} \\
    			\vec{c}^{T} 
    		}
    		\vec{d}
    		=
    		\myvec{
    			0 \\ 0 \\ 15
    		}
    	\end{align}
    	
    
     
Numerically, using $\vec{a}$, $\vec{b}$, $\vec{c}$ the above equation \ref{eq:solution_point_vector_15_6} can be written as,

     
\begin{align}
	\myvec{
		1 & 4 & 2 \\
		3 & -2 & 7 \\
		2 & -1 & 4 
	}
	\vec{d}
	=
	\myvec{
		0 \\ 0 \\ 15
	}
\end{align}

    
    we can use Guassian Elimination Method in order to find the coordinate values of $\vec{d}$.
    
    

	\begin{align}
		\myvec{
			1 & 4 & 2 & \vrule & 0 \\
			3 & -2 & 7 & \vrule & 0 \\
			2 & -1 & 4 & \vrule & 15 \\
		}
		\\
		\xleftrightarrow[R_2 \leftarrow R_2-3R_1]{R_3 \leftarrow R_3 - 2R_1}
		\myvec{
			1 & 4 & 2 & \vrule & 0 \\
			0 & -14 & 1 & \vrule & 0 \\
			0 & -9 & 0 & \vrule & 15
		}
		\\
		\xleftrightarrow[]{R_3\leftarrow R_3-\frac{9}{14}R_2}
		\myvec{
			1 & 4 & 2 & \vrule & 0 \\
			0 & -14 & 1 & \vrule & 0 \\
			0 & 0 & \frac{-9}{14} & \vrule & 15
		}\\
		\xleftrightarrow[R_2 \leftarrow \frac{-1}{14}R_2]{R_3 \leftarrow \frac{-14}{9}R_2}
		\myvec{
			1 & 4 & 2 & \vrule & 0 \\[0.2cm]
			0 & 1 & \frac{-1}{14} & \vrule & 0 \\[0.2cm]
			0 & 0 & 1 & \vrule & \frac{-210}{9}
		}\\
		\xleftrightarrow[]{R_1 \leftarrow R_1+\frac{1}{14}R_3}
		\myvec{
			1 & 4 & 2 & \vrule & 0 \\[0.2cm]
			0 & 1 & 0 & \vrule & \frac{-210}{126} \\[0.2cm]
			0 & 0 & 1 & \vrule & \frac{-210}{9}
		}
	\end{align}


	\begin{align}
		\xleftrightarrow[]{R_1 \leftarrow R_1-4R_3}
		\myvec{
			1 & 0 & 2 & \vrule & \frac{840}{126} \\[0.2cm]
			0 & 1 & 0 & \vrule & \frac{-210}{126} \\[0.2cm]
			0 & 0 & 1 & \vrule & \frac{-210}{9}
		}\\
		\xleftrightarrow[]{R_1 \leftarrow R_1-2R_3}
		\myvec{
			1 & 0 & 0 & \vrule & \frac{6720}{126} \\[0.2cm]
			0 & 1 & 0 & \vrule & \frac{-210}{126} \\[0.2cm]
			0 & 0 & 1 & \vrule & \frac{-210}{9}
		}
\end{align}


By using Guassian Elimination Method, we were able to get the vector $\vec{d}$ as
	$\myvec{\frac{6720}{126} \\[0.2cm]\frac{-210}{126} \\[0.2cm]\frac{-210}{9}}$\\[0.5cm]
	
