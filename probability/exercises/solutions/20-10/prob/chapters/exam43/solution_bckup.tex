Given that a black ball is selected, the probability that it is picked from box III is the case of conditional probability expressed as
\begin{align}
\pr{III|B}=\frac{\pr{III \cup B}}{\pr{B}}
\label{eq:exam43_3}
\end{align}
By the definition of conditional probability
\begin{align}
\pr{B|III}=\frac{\pr{III \cup B}}{\pr{III}}
\\
\pr{III \cup B} = \pr{B|III}\pr{III}
\label{eq:exam43_1}
\end{align}
Also 
\begin{align}
\implies \pr{B}=\pr{I \cup B}+\pr{II \cup B}
\\ +\pr{III \cup B}+\pr{IV \cup B}
\\
\implies \pr{B} = \pr{B|I} \pr{I}+\pr{B|II} \pr{II}+
\\ \pr{B|III} \pr{III}+\pr{B|IV} \pr{IV}
\label{eq:exam43_2}
\end{align}

Substituting \eqref{eq:exam43_1} and \eqref{eq:exam43_2} in \eqref{eq:exam43_3}, We obtain the Baye's theorm as stated in \eqref{eq:exam43_bayestheorm}
\begin{align}
\pr{III|B}= \frac{\pr{B|III} \pr{III}}{\pr{B|I} \pr{I}+\pr{B|II} \pr{II}+\pr{B|III} \pr{III}+\pr{B|IV} \pr{IV}}
\label{eq:exam43_bayestheorm}
\end{align}
From the given data,
\begin{align}
\pr{B|I} = \frac{1}{6}
\\
\pr{B|II} = \frac{1}{4}
\\
\pr{B|III} = \frac{1}{7}
\\
\pr{B|IV} = \frac{4}{13}
\\
\pr{I}=\pr{II}=\pr{III}=\pr{IV}
\end{align}
Substituting the above values in equation \eqref{eq:exam43_bayestheorm},
\begin{align}
\pr{III|B}=\frac{156}{947}
\end{align}
The python code for the above problem is,
\begin{lstlisting}
./prob/codes/exam43.py
\end{lstlisting}
