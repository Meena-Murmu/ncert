Two events E and F are said to be independent if they satisfy the criterion:
\begin{align}
P(E \cap F) &= P(E)P(F)
\end{align}
\begin{enumerate}
\item There are 13 cards of spades, 4 cards of aces and 1 card of ace of spades.
\begin{align}
P(E) &= \frac{13}{52}\\
P(F) &= \frac{4}{52}\\
P(E \cap F) &= \frac{1}{52}
\end{align}
Clearly, $P(E \cap F)= P(E)P(F)$. Therefore E and F are independent events.
\item There are 26 black cards, 4 king cards and 2 black and king cards.
\begin{align}
P(E) &= \frac{26}{52}\\
P(F) &= \frac{4}{52}\\
P(E \cap F) &= \frac{2}{52}
\end{align}
Clearly, $P(E \cap F)= P(E)P(F)$. Therefore E and F are independent events.
\item There are 8 kings or queens, 8 queens or jacks. In both of these, common is the quuen cards. 
\begin{align}
P(E) &= \frac{8}{52}\\
P(F) &= \frac{8}{52}\\
P(E \cap F) &= \frac{4}{52}
\end{align}
Clearly, $P(E \cap F) \neq P(E)P(F)$. Therefore E and F are not independent events.
\end{enumerate}

