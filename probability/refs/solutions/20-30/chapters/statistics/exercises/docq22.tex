   The following python code computes the mean, median and mode.
	\begin{lstlisting}
	codes/statistics/exercises/q22.py
	\end{lstlisting}
	\begin{align}
	\text{Median} &= l + \frac{\frac{n}{2} -cf}{f}\times h
	\end{align}
	\begin{align}
	\text{n} = \sum f_{i} = 100 \implies \frac{n}{2} = 50\\
	\end{align}
	$\therefore$ 55-60 is the median class.\\
	Here l is the lower limit of the median class = 55\\
	h is the classinterval =5\\
	cf is the cumulative frequency of the class before median class = 13\\
	f is the frequency of the median class

	\begin{align}
	\text{Median} &= 55 + \frac{15 - 13}{6}\times 5\\
	\text{Median} &= 55 + 1.67 = 56.67
	\end{align}
	Hence median weight is 56.67

